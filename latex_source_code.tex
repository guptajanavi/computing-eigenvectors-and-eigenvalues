\documentclass[11pt]{article}

    \usepackage[breakable]{tcolorbox}
    \usepackage{parskip} % Stop auto-indenting (to mimic markdown behaviour)
    

    % Basic figure setup, for now with no caption control since it's done
    % automatically by Pandoc (which extracts ![](path) syntax from Markdown).
    \usepackage{graphicx}
    % Maintain compatibility with old templates. Remove in nbconvert 6.0
    \let\Oldincludegraphics\includegraphics
    % Ensure that by default, figures have no caption (until we provide a
    % proper Figure object with a Caption API and a way to capture that
    % in the conversion process - todo).
    \usepackage{caption}
    \DeclareCaptionFormat{nocaption}{}
    \captionsetup{format=nocaption,aboveskip=0pt,belowskip=0pt}

    \usepackage{float}
    \floatplacement{figure}{H} % forces figures to be placed at the correct location
    \usepackage{xcolor} % Allow colors to be defined
    \usepackage{enumerate} % Needed for markdown enumerations to work
    \usepackage{geometry} % Used to adjust the document margins
    \usepackage{amsmath} % Equations
    \usepackage{amssymb} % Equations
    \usepackage{textcomp} % defines textquotesingle
    % Hack from http://tex.stackexchange.com/a/47451/13684:
    \AtBeginDocument{%
        \def\PYZsq{\textquotesingle}% Upright quotes in Pygmentized code
    }
    \usepackage{upquote} % Upright quotes for verbatim code
    \usepackage{eurosym} % defines \euro

    \usepackage{iftex}
    \ifPDFTeX
        \usepackage[T1]{fontenc}
        \IfFileExists{alphabeta.sty}{
              \usepackage{alphabeta}
          }{
              \usepackage[mathletters]{ucs}
              \usepackage[utf8x]{inputenc}
          }
    \else
        \usepackage{fontspec}
        \usepackage{unicode-math}
    \fi

    \usepackage{fancyvrb} % verbatim replacement that allows latex
    \usepackage{grffile} % extends the file name processing of package graphics
                         % to support a larger range
    \makeatletter % fix for old versions of grffile with XeLaTeX
    \@ifpackagelater{grffile}{2019/11/01}
    {
      % Do nothing on new versions
    }
    {
      \def\Gread@@xetex#1{%
        \IfFileExists{"\Gin@base".bb}%
        {\Gread@eps{\Gin@base.bb}}%
        {\Gread@@xetex@aux#1}%
      }
    }
    \makeatother
    \usepackage[Export]{adjustbox} % Used to constrain images to a maximum size
    \adjustboxset{max size={0.9\linewidth}{0.9\paperheight}}

    % The hyperref package gives us a pdf with properly built
    % internal navigation ('pdf bookmarks' for the table of contents,
    % internal cross-reference links, web links for URLs, etc.)
    \usepackage{hyperref}
    % The default LaTeX title has an obnoxious amount of whitespace. By default,
    % titling removes some of it. It also provides customization options.
    \usepackage{titling}
    \usepackage{longtable} % longtable support required by pandoc >1.10
    \usepackage{booktabs}  % table support for pandoc > 1.12.2
    \usepackage{array}     % table support for pandoc >= 2.11.3
    \usepackage{calc}      % table minipage width calculation for pandoc >= 2.11.1
    \usepackage[inline]{enumitem} % IRkernel/repr support (it uses the enumerate* environment)
    \usepackage[normalem]{ulem} % ulem is needed to support strikethroughs (\sout)
                                % normalem makes italics be italics, not underlines
    \usepackage{soul}      % strikethrough (\st) support for pandoc >= 3.0.0
    \usepackage{mathrsfs}
    

    
    % Colors for the hyperref package
    \definecolor{urlcolor}{rgb}{0,.145,.698}
    \definecolor{linkcolor}{rgb}{.71,0.21,0.01}
    \definecolor{citecolor}{rgb}{.12,.54,.11}

    % ANSI colors
    \definecolor{ansi-black}{HTML}{3E424D}
    \definecolor{ansi-black-intense}{HTML}{282C36}
    \definecolor{ansi-red}{HTML}{E75C58}
    \definecolor{ansi-red-intense}{HTML}{B22B31}
    \definecolor{ansi-green}{HTML}{00A250}
    \definecolor{ansi-green-intense}{HTML}{007427}
    \definecolor{ansi-yellow}{HTML}{DDB62B}
    \definecolor{ansi-yellow-intense}{HTML}{B27D12}
    \definecolor{ansi-blue}{HTML}{208FFB}
    \definecolor{ansi-blue-intense}{HTML}{0065CA}
    \definecolor{ansi-magenta}{HTML}{D160C4}
    \definecolor{ansi-magenta-intense}{HTML}{A03196}
    \definecolor{ansi-cyan}{HTML}{60C6C8}
    \definecolor{ansi-cyan-intense}{HTML}{258F8F}
    \definecolor{ansi-white}{HTML}{C5C1B4}
    \definecolor{ansi-white-intense}{HTML}{A1A6B2}
    \definecolor{ansi-default-inverse-fg}{HTML}{FFFFFF}
    \definecolor{ansi-default-inverse-bg}{HTML}{000000}

    % common color for the border for error outputs.
    \definecolor{outerrorbackground}{HTML}{FFDFDF}

    % commands and environments needed by pandoc snippets
    % extracted from the output of `pandoc -s`
    \providecommand{\tightlist}{%
      \setlength{\itemsep}{0pt}\setlength{\parskip}{0pt}}
    \DefineVerbatimEnvironment{Highlighting}{Verbatim}{commandchars=\\\{\}}
    % Add ',fontsize=\small' for more characters per line
    \newenvironment{Shaded}{}{}
    \newcommand{\KeywordTok}[1]{\textcolor[rgb]{0.00,0.44,0.13}{\textbf{{#1}}}}
    \newcommand{\DataTypeTok}[1]{\textcolor[rgb]{0.56,0.13,0.00}{{#1}}}
    \newcommand{\DecValTok}[1]{\textcolor[rgb]{0.25,0.63,0.44}{{#1}}}
    \newcommand{\BaseNTok}[1]{\textcolor[rgb]{0.25,0.63,0.44}{{#1}}}
    \newcommand{\FloatTok}[1]{\textcolor[rgb]{0.25,0.63,0.44}{{#1}}}
    \newcommand{\CharTok}[1]{\textcolor[rgb]{0.25,0.44,0.63}{{#1}}}
    \newcommand{\StringTok}[1]{\textcolor[rgb]{0.25,0.44,0.63}{{#1}}}
    \newcommand{\CommentTok}[1]{\textcolor[rgb]{0.38,0.63,0.69}{\textit{{#1}}}}
    \newcommand{\OtherTok}[1]{\textcolor[rgb]{0.00,0.44,0.13}{{#1}}}
    \newcommand{\AlertTok}[1]{\textcolor[rgb]{1.00,0.00,0.00}{\textbf{{#1}}}}
    \newcommand{\FunctionTok}[1]{\textcolor[rgb]{0.02,0.16,0.49}{{#1}}}
    \newcommand{\RegionMarkerTok}[1]{{#1}}
    \newcommand{\ErrorTok}[1]{\textcolor[rgb]{1.00,0.00,0.00}{\textbf{{#1}}}}
    \newcommand{\NormalTok}[1]{{#1}}

    % Additional commands for more recent versions of Pandoc
    \newcommand{\ConstantTok}[1]{\textcolor[rgb]{0.53,0.00,0.00}{{#1}}}
    \newcommand{\SpecialCharTok}[1]{\textcolor[rgb]{0.25,0.44,0.63}{{#1}}}
    \newcommand{\VerbatimStringTok}[1]{\textcolor[rgb]{0.25,0.44,0.63}{{#1}}}
    \newcommand{\SpecialStringTok}[1]{\textcolor[rgb]{0.73,0.40,0.53}{{#1}}}
    \newcommand{\ImportTok}[1]{{#1}}
    \newcommand{\DocumentationTok}[1]{\textcolor[rgb]{0.73,0.13,0.13}{\textit{{#1}}}}
    \newcommand{\AnnotationTok}[1]{\textcolor[rgb]{0.38,0.63,0.69}{\textbf{\textit{{#1}}}}}
    \newcommand{\CommentVarTok}[1]{\textcolor[rgb]{0.38,0.63,0.69}{\textbf{\textit{{#1}}}}}
    \newcommand{\VariableTok}[1]{\textcolor[rgb]{0.10,0.09,0.49}{{#1}}}
    \newcommand{\ControlFlowTok}[1]{\textcolor[rgb]{0.00,0.44,0.13}{\textbf{{#1}}}}
    \newcommand{\OperatorTok}[1]{\textcolor[rgb]{0.40,0.40,0.40}{{#1}}}
    \newcommand{\BuiltInTok}[1]{{#1}}
    \newcommand{\ExtensionTok}[1]{{#1}}
    \newcommand{\PreprocessorTok}[1]{\textcolor[rgb]{0.74,0.48,0.00}{{#1}}}
    \newcommand{\AttributeTok}[1]{\textcolor[rgb]{0.49,0.56,0.16}{{#1}}}
    \newcommand{\InformationTok}[1]{\textcolor[rgb]{0.38,0.63,0.69}{\textbf{\textit{{#1}}}}}
    \newcommand{\WarningTok}[1]{\textcolor[rgb]{0.38,0.63,0.69}{\textbf{\textit{{#1}}}}}


    % Define a nice break command that doesn't care if a line doesn't already
    % exist.
    \def\br{\hspace*{\fill} \\* }
    % Math Jax compatibility definitions
    \def\gt{>}
    \def\lt{<}
    \let\Oldtex\TeX
    \let\Oldlatex\LaTeX
    \renewcommand{\TeX}{\textrm{\Oldtex}}
    \renewcommand{\LaTeX}{\textrm{\Oldlatex}}
    % Document parameters
    % Document title
    \title{21241\_final\_project}
    
    
    
    
    
    
    
% Pygments definitions
\makeatletter
\def\PY@reset{\let\PY@it=\relax \let\PY@bf=\relax%
    \let\PY@ul=\relax \let\PY@tc=\relax%
    \let\PY@bc=\relax \let\PY@ff=\relax}
\def\PY@tok#1{\csname PY@tok@#1\endcsname}
\def\PY@toks#1+{\ifx\relax#1\empty\else%
    \PY@tok{#1}\expandafter\PY@toks\fi}
\def\PY@do#1{\PY@bc{\PY@tc{\PY@ul{%
    \PY@it{\PY@bf{\PY@ff{#1}}}}}}}
\def\PY#1#2{\PY@reset\PY@toks#1+\relax+\PY@do{#2}}

\@namedef{PY@tok@w}{\def\PY@tc##1{\textcolor[rgb]{0.73,0.73,0.73}{##1}}}
\@namedef{PY@tok@c}{\let\PY@it=\textit\def\PY@tc##1{\textcolor[rgb]{0.24,0.48,0.48}{##1}}}
\@namedef{PY@tok@cp}{\def\PY@tc##1{\textcolor[rgb]{0.61,0.40,0.00}{##1}}}
\@namedef{PY@tok@k}{\let\PY@bf=\textbf\def\PY@tc##1{\textcolor[rgb]{0.00,0.50,0.00}{##1}}}
\@namedef{PY@tok@kp}{\def\PY@tc##1{\textcolor[rgb]{0.00,0.50,0.00}{##1}}}
\@namedef{PY@tok@kt}{\def\PY@tc##1{\textcolor[rgb]{0.69,0.00,0.25}{##1}}}
\@namedef{PY@tok@o}{\def\PY@tc##1{\textcolor[rgb]{0.40,0.40,0.40}{##1}}}
\@namedef{PY@tok@ow}{\let\PY@bf=\textbf\def\PY@tc##1{\textcolor[rgb]{0.67,0.13,1.00}{##1}}}
\@namedef{PY@tok@nb}{\def\PY@tc##1{\textcolor[rgb]{0.00,0.50,0.00}{##1}}}
\@namedef{PY@tok@nf}{\def\PY@tc##1{\textcolor[rgb]{0.00,0.00,1.00}{##1}}}
\@namedef{PY@tok@nc}{\let\PY@bf=\textbf\def\PY@tc##1{\textcolor[rgb]{0.00,0.00,1.00}{##1}}}
\@namedef{PY@tok@nn}{\let\PY@bf=\textbf\def\PY@tc##1{\textcolor[rgb]{0.00,0.00,1.00}{##1}}}
\@namedef{PY@tok@ne}{\let\PY@bf=\textbf\def\PY@tc##1{\textcolor[rgb]{0.80,0.25,0.22}{##1}}}
\@namedef{PY@tok@nv}{\def\PY@tc##1{\textcolor[rgb]{0.10,0.09,0.49}{##1}}}
\@namedef{PY@tok@no}{\def\PY@tc##1{\textcolor[rgb]{0.53,0.00,0.00}{##1}}}
\@namedef{PY@tok@nl}{\def\PY@tc##1{\textcolor[rgb]{0.46,0.46,0.00}{##1}}}
\@namedef{PY@tok@ni}{\let\PY@bf=\textbf\def\PY@tc##1{\textcolor[rgb]{0.44,0.44,0.44}{##1}}}
\@namedef{PY@tok@na}{\def\PY@tc##1{\textcolor[rgb]{0.41,0.47,0.13}{##1}}}
\@namedef{PY@tok@nt}{\let\PY@bf=\textbf\def\PY@tc##1{\textcolor[rgb]{0.00,0.50,0.00}{##1}}}
\@namedef{PY@tok@nd}{\def\PY@tc##1{\textcolor[rgb]{0.67,0.13,1.00}{##1}}}
\@namedef{PY@tok@s}{\def\PY@tc##1{\textcolor[rgb]{0.73,0.13,0.13}{##1}}}
\@namedef{PY@tok@sd}{\let\PY@it=\textit\def\PY@tc##1{\textcolor[rgb]{0.73,0.13,0.13}{##1}}}
\@namedef{PY@tok@si}{\let\PY@bf=\textbf\def\PY@tc##1{\textcolor[rgb]{0.64,0.35,0.47}{##1}}}
\@namedef{PY@tok@se}{\let\PY@bf=\textbf\def\PY@tc##1{\textcolor[rgb]{0.67,0.36,0.12}{##1}}}
\@namedef{PY@tok@sr}{\def\PY@tc##1{\textcolor[rgb]{0.64,0.35,0.47}{##1}}}
\@namedef{PY@tok@ss}{\def\PY@tc##1{\textcolor[rgb]{0.10,0.09,0.49}{##1}}}
\@namedef{PY@tok@sx}{\def\PY@tc##1{\textcolor[rgb]{0.00,0.50,0.00}{##1}}}
\@namedef{PY@tok@m}{\def\PY@tc##1{\textcolor[rgb]{0.40,0.40,0.40}{##1}}}
\@namedef{PY@tok@gh}{\let\PY@bf=\textbf\def\PY@tc##1{\textcolor[rgb]{0.00,0.00,0.50}{##1}}}
\@namedef{PY@tok@gu}{\let\PY@bf=\textbf\def\PY@tc##1{\textcolor[rgb]{0.50,0.00,0.50}{##1}}}
\@namedef{PY@tok@gd}{\def\PY@tc##1{\textcolor[rgb]{0.63,0.00,0.00}{##1}}}
\@namedef{PY@tok@gi}{\def\PY@tc##1{\textcolor[rgb]{0.00,0.52,0.00}{##1}}}
\@namedef{PY@tok@gr}{\def\PY@tc##1{\textcolor[rgb]{0.89,0.00,0.00}{##1}}}
\@namedef{PY@tok@ge}{\let\PY@it=\textit}
\@namedef{PY@tok@gs}{\let\PY@bf=\textbf}
\@namedef{PY@tok@ges}{\let\PY@bf=\textbf\let\PY@it=\textit}
\@namedef{PY@tok@gp}{\let\PY@bf=\textbf\def\PY@tc##1{\textcolor[rgb]{0.00,0.00,0.50}{##1}}}
\@namedef{PY@tok@go}{\def\PY@tc##1{\textcolor[rgb]{0.44,0.44,0.44}{##1}}}
\@namedef{PY@tok@gt}{\def\PY@tc##1{\textcolor[rgb]{0.00,0.27,0.87}{##1}}}
\@namedef{PY@tok@err}{\def\PY@bc##1{{\setlength{\fboxsep}{\string -\fboxrule}\fcolorbox[rgb]{1.00,0.00,0.00}{1,1,1}{\strut ##1}}}}
\@namedef{PY@tok@kc}{\let\PY@bf=\textbf\def\PY@tc##1{\textcolor[rgb]{0.00,0.50,0.00}{##1}}}
\@namedef{PY@tok@kd}{\let\PY@bf=\textbf\def\PY@tc##1{\textcolor[rgb]{0.00,0.50,0.00}{##1}}}
\@namedef{PY@tok@kn}{\let\PY@bf=\textbf\def\PY@tc##1{\textcolor[rgb]{0.00,0.50,0.00}{##1}}}
\@namedef{PY@tok@kr}{\let\PY@bf=\textbf\def\PY@tc##1{\textcolor[rgb]{0.00,0.50,0.00}{##1}}}
\@namedef{PY@tok@bp}{\def\PY@tc##1{\textcolor[rgb]{0.00,0.50,0.00}{##1}}}
\@namedef{PY@tok@fm}{\def\PY@tc##1{\textcolor[rgb]{0.00,0.00,1.00}{##1}}}
\@namedef{PY@tok@vc}{\def\PY@tc##1{\textcolor[rgb]{0.10,0.09,0.49}{##1}}}
\@namedef{PY@tok@vg}{\def\PY@tc##1{\textcolor[rgb]{0.10,0.09,0.49}{##1}}}
\@namedef{PY@tok@vi}{\def\PY@tc##1{\textcolor[rgb]{0.10,0.09,0.49}{##1}}}
\@namedef{PY@tok@vm}{\def\PY@tc##1{\textcolor[rgb]{0.10,0.09,0.49}{##1}}}
\@namedef{PY@tok@sa}{\def\PY@tc##1{\textcolor[rgb]{0.73,0.13,0.13}{##1}}}
\@namedef{PY@tok@sb}{\def\PY@tc##1{\textcolor[rgb]{0.73,0.13,0.13}{##1}}}
\@namedef{PY@tok@sc}{\def\PY@tc##1{\textcolor[rgb]{0.73,0.13,0.13}{##1}}}
\@namedef{PY@tok@dl}{\def\PY@tc##1{\textcolor[rgb]{0.73,0.13,0.13}{##1}}}
\@namedef{PY@tok@s2}{\def\PY@tc##1{\textcolor[rgb]{0.73,0.13,0.13}{##1}}}
\@namedef{PY@tok@sh}{\def\PY@tc##1{\textcolor[rgb]{0.73,0.13,0.13}{##1}}}
\@namedef{PY@tok@s1}{\def\PY@tc##1{\textcolor[rgb]{0.73,0.13,0.13}{##1}}}
\@namedef{PY@tok@mb}{\def\PY@tc##1{\textcolor[rgb]{0.40,0.40,0.40}{##1}}}
\@namedef{PY@tok@mf}{\def\PY@tc##1{\textcolor[rgb]{0.40,0.40,0.40}{##1}}}
\@namedef{PY@tok@mh}{\def\PY@tc##1{\textcolor[rgb]{0.40,0.40,0.40}{##1}}}
\@namedef{PY@tok@mi}{\def\PY@tc##1{\textcolor[rgb]{0.40,0.40,0.40}{##1}}}
\@namedef{PY@tok@il}{\def\PY@tc##1{\textcolor[rgb]{0.40,0.40,0.40}{##1}}}
\@namedef{PY@tok@mo}{\def\PY@tc##1{\textcolor[rgb]{0.40,0.40,0.40}{##1}}}
\@namedef{PY@tok@ch}{\let\PY@it=\textit\def\PY@tc##1{\textcolor[rgb]{0.24,0.48,0.48}{##1}}}
\@namedef{PY@tok@cm}{\let\PY@it=\textit\def\PY@tc##1{\textcolor[rgb]{0.24,0.48,0.48}{##1}}}
\@namedef{PY@tok@cpf}{\let\PY@it=\textit\def\PY@tc##1{\textcolor[rgb]{0.24,0.48,0.48}{##1}}}
\@namedef{PY@tok@c1}{\let\PY@it=\textit\def\PY@tc##1{\textcolor[rgb]{0.24,0.48,0.48}{##1}}}
\@namedef{PY@tok@cs}{\let\PY@it=\textit\def\PY@tc##1{\textcolor[rgb]{0.24,0.48,0.48}{##1}}}

\def\PYZbs{\char`\\}
\def\PYZus{\char`\_}
\def\PYZob{\char`\{}
\def\PYZcb{\char`\}}
\def\PYZca{\char`\^}
\def\PYZam{\char`\&}
\def\PYZlt{\char`\<}
\def\PYZgt{\char`\>}
\def\PYZsh{\char`\#}
\def\PYZpc{\char`\%}
\def\PYZdl{\char`\$}
\def\PYZhy{\char`\-}
\def\PYZsq{\char`\'}
\def\PYZdq{\char`\"}
\def\PYZti{\char`\~}
% for compatibility with earlier versions
\def\PYZat{@}
\def\PYZlb{[}
\def\PYZrb{]}
\makeatother


    % For linebreaks inside Verbatim environment from package fancyvrb.
    \makeatletter
        \newbox\Wrappedcontinuationbox
        \newbox\Wrappedvisiblespacebox
        \newcommand*\Wrappedvisiblespace {\textcolor{red}{\textvisiblespace}}
        \newcommand*\Wrappedcontinuationsymbol {\textcolor{red}{\llap{\tiny$\m@th\hookrightarrow$}}}
        \newcommand*\Wrappedcontinuationindent {3ex }
        \newcommand*\Wrappedafterbreak {\kern\Wrappedcontinuationindent\copy\Wrappedcontinuationbox}
        % Take advantage of the already applied Pygments mark-up to insert
        % potential linebreaks for TeX processing.
        %        {, <, #, %, $, ' and ": go to next line.
        %        _, }, ^, &, >, - and ~: stay at end of broken line.
        % Use of \textquotesingle for straight quote.
        \newcommand*\Wrappedbreaksatspecials {%
            \def\PYGZus{\discretionary{\char`\_}{\Wrappedafterbreak}{\char`\_}}%
            \def\PYGZob{\discretionary{}{\Wrappedafterbreak\char`\{}{\char`\{}}%
            \def\PYGZcb{\discretionary{\char`\}}{\Wrappedafterbreak}{\char`\}}}%
            \def\PYGZca{\discretionary{\char`\^}{\Wrappedafterbreak}{\char`\^}}%
            \def\PYGZam{\discretionary{\char`\&}{\Wrappedafterbreak}{\char`\&}}%
            \def\PYGZlt{\discretionary{}{\Wrappedafterbreak\char`\<}{\char`\<}}%
            \def\PYGZgt{\discretionary{\char`\>}{\Wrappedafterbreak}{\char`\>}}%
            \def\PYGZsh{\discretionary{}{\Wrappedafterbreak\char`\#}{\char`\#}}%
            \def\PYGZpc{\discretionary{}{\Wrappedafterbreak\char`\%}{\char`\%}}%
            \def\PYGZdl{\discretionary{}{\Wrappedafterbreak\char`\$}{\char`\$}}%
            \def\PYGZhy{\discretionary{\char`\-}{\Wrappedafterbreak}{\char`\-}}%
            \def\PYGZsq{\discretionary{}{\Wrappedafterbreak\textquotesingle}{\textquotesingle}}%
            \def\PYGZdq{\discretionary{}{\Wrappedafterbreak\char`\"}{\char`\"}}%
            \def\PYGZti{\discretionary{\char`\~}{\Wrappedafterbreak}{\char`\~}}%
        }
        % Some characters . , ; ? ! / are not pygmentized.
        % This macro makes them "active" and they will insert potential linebreaks
        \newcommand*\Wrappedbreaksatpunct {%
            \lccode`\~`\.\lowercase{\def~}{\discretionary{\hbox{\char`\.}}{\Wrappedafterbreak}{\hbox{\char`\.}}}%
            \lccode`\~`\,\lowercase{\def~}{\discretionary{\hbox{\char`\,}}{\Wrappedafterbreak}{\hbox{\char`\,}}}%
            \lccode`\~`\;\lowercase{\def~}{\discretionary{\hbox{\char`\;}}{\Wrappedafterbreak}{\hbox{\char`\;}}}%
            \lccode`\~`\:\lowercase{\def~}{\discretionary{\hbox{\char`\:}}{\Wrappedafterbreak}{\hbox{\char`\:}}}%
            \lccode`\~`\?\lowercase{\def~}{\discretionary{\hbox{\char`\?}}{\Wrappedafterbreak}{\hbox{\char`\?}}}%
            \lccode`\~`\!\lowercase{\def~}{\discretionary{\hbox{\char`\!}}{\Wrappedafterbreak}{\hbox{\char`\!}}}%
            \lccode`\~`\/\lowercase{\def~}{\discretionary{\hbox{\char`\/}}{\Wrappedafterbreak}{\hbox{\char`\/}}}%
            \catcode`\.\active
            \catcode`\,\active
            \catcode`\;\active
            \catcode`\:\active
            \catcode`\?\active
            \catcode`\!\active
            \catcode`\/\active
            \lccode`\~`\~
        }
    \makeatother

    \let\OriginalVerbatim=\Verbatim
    \makeatletter
    \renewcommand{\Verbatim}[1][1]{%
        %\parskip\z@skip
        \sbox\Wrappedcontinuationbox {\Wrappedcontinuationsymbol}%
        \sbox\Wrappedvisiblespacebox {\FV@SetupFont\Wrappedvisiblespace}%
        \def\FancyVerbFormatLine ##1{\hsize\linewidth
            \vtop{\raggedright\hyphenpenalty\z@\exhyphenpenalty\z@
                \doublehyphendemerits\z@\finalhyphendemerits\z@
                \strut ##1\strut}%
        }%
        % If the linebreak is at a space, the latter will be displayed as visible
        % space at end of first line, and a continuation symbol starts next line.
        % Stretch/shrink are however usually zero for typewriter font.
        \def\FV@Space {%
            \nobreak\hskip\z@ plus\fontdimen3\font minus\fontdimen4\font
            \discretionary{\copy\Wrappedvisiblespacebox}{\Wrappedafterbreak}
            {\kern\fontdimen2\font}%
        }%

        % Allow breaks at special characters using \PYG... macros.
        \Wrappedbreaksatspecials
        % Breaks at punctuation characters . , ; ? ! and / need catcode=\active
        \OriginalVerbatim[#1,codes*=\Wrappedbreaksatpunct]%
    }
    \makeatother

    % Exact colors from NB
    \definecolor{incolor}{HTML}{303F9F}
    \definecolor{outcolor}{HTML}{D84315}
    \definecolor{cellborder}{HTML}{CFCFCF}
    \definecolor{cellbackground}{HTML}{F7F7F7}

    % prompt
    \makeatletter
    \newcommand{\boxspacing}{\kern\kvtcb@left@rule\kern\kvtcb@boxsep}
    \makeatother
    \newcommand{\prompt}[4]{
        {\ttfamily\llap{{\color{#2}[#3]:\hspace{3pt}#4}}\vspace{-\baselineskip}}
    }
    

    
    % Prevent overflowing lines due to hard-to-break entities
    \sloppy
    % Setup hyperref package
    \hypersetup{
      breaklinks=true,  % so long urls are correctly broken across lines
      colorlinks=true,
      urlcolor=urlcolor,
      linkcolor=linkcolor,
      citecolor=citecolor,
      }
    % Slightly bigger margins than the latex defaults
    
    \geometry{verbose,tmargin=1in,bmargin=1in,lmargin=1in,rmargin=1in}
    
    

\begin{document}
    
    \maketitle
    
    

    
    \begin{tcolorbox}[breakable, size=fbox, boxrule=1pt, pad at break*=1mm,colback=cellbackground, colframe=cellborder]
\prompt{In}{incolor}{131}{\boxspacing}
\begin{Verbatim}[commandchars=\\\{\}]
\PY{k}{using}\PY{+w}{ }\PY{n}{LinearAlgebra}
\PY{k}{using}\PY{+w}{ }\PY{n}{PrettyTables}
\end{Verbatim}
\end{tcolorbox}

    \begin{tcolorbox}[breakable, size=fbox, boxrule=1pt, pad at break*=1mm,colback=cellbackground, colframe=cellborder]
\prompt{In}{incolor}{132}{\boxspacing}
\begin{Verbatim}[commandchars=\\\{\}]
\PY{c}{\PYZsh{} Function to approximate the leading eigenvalue of a matrix using the QR method}
\PY{k}{function}\PY{+w}{ }\PY{n}{qr\PYZus{}method}\PY{p}{(}\PY{n}{A}\PY{p}{,}\PY{+w}{ }\PY{n}{max\PYZus{}iter}\PY{p}{)}
\PY{+w}{    }\PY{c}{\PYZsh{} Initialize A\PYZus{}k with the input matrix A}
\PY{+w}{    }\PY{n}{A\PYZus{}k}\PY{+w}{ }\PY{o}{=}\PY{+w}{ }\PY{n}{copy}\PY{p}{(}\PY{n}{A}\PY{p}{)}

\PY{+w}{    }\PY{c}{\PYZsh{} Iterate up to a maximum number of iterations}
\PY{+w}{    }\PY{k}{for}\PY{+w}{ }\PY{n}{k}\PY{+w}{ }\PY{o}{=}\PY{+w}{ }\PY{l+m+mi}{1}\PY{o}{:}\PY{n}{max\PYZus{}iter}
\PY{+w}{        }\PY{c}{\PYZsh{} Perform QR decomposition on A\PYZus{}k}
\PY{+w}{        }\PY{n}{Q}\PY{p}{,}\PY{+w}{ }\PY{n}{R}\PY{+w}{ }\PY{o}{=}\PY{+w}{ }\PY{n}{qr}\PY{p}{(}\PY{n}{A\PYZus{}k}\PY{p}{)}
\PY{+w}{        }\PY{c}{\PYZsh{} Form the next matrix A\PYZus{}k by multiplying R and Q}
\PY{+w}{        }\PY{n}{A\PYZus{}k}\PY{+w}{ }\PY{o}{=}\PY{+w}{ }\PY{n}{R}\PY{+w}{ }\PY{o}{*}\PY{+w}{ }\PY{n}{Q}
\PY{+w}{    }\PY{k}{end}

\PY{+w}{    }\PY{c}{\PYZsh{} \PYZsh{} Return the leading eigenvalue, which is the first diagonal element of A\PYZus{}k}
\PY{+w}{    }\PY{k}{return}\PY{+w}{ }\PY{n}{A\PYZus{}k}\PY{p}{[}\PY{l+m+mi}{1}\PY{p}{,}\PY{l+m+mi}{1}\PY{p}{]}
\PY{k}{end}
\end{Verbatim}
\end{tcolorbox}

            \begin{tcolorbox}[breakable, size=fbox, boxrule=.5pt, pad at break*=1mm, opacityfill=0]
\prompt{Out}{outcolor}{132}{\boxspacing}
\begin{Verbatim}[commandchars=\\\{\}]
qr\_method (generic function with 1 method)
\end{Verbatim}
\end{tcolorbox}
        
    \begin{tcolorbox}[breakable, size=fbox, boxrule=1pt, pad at break*=1mm,colback=cellbackground, colframe=cellborder]
\prompt{In}{incolor}{133}{\boxspacing}
\begin{Verbatim}[commandchars=\\\{\}]
\PY{c}{\PYZsh{} Define a function to compute the leading eigenvalue using the Power Method}
\PY{k}{function}\PY{+w}{ }\PY{n}{power\PYZus{}method}\PY{p}{(}\PY{n}{A}\PY{p}{,}\PY{+w}{ }\PY{n}{max\PYZus{}iter}\PY{p}{)}
\PY{+w}{    }\PY{c}{\PYZsh{} Determine the size of the matrix A and initialize a random vector of this size }
\PY{+w}{    }\PY{n}{n}\PY{p}{,}\PY{+w}{ }\PY{n}{\PYZus{}}\PY{+w}{ }\PY{o}{=}\PY{+w}{ }\PY{n}{size}\PY{p}{(}\PY{n}{A}\PY{p}{)}
\PY{+w}{    }\PY{n}{v}\PY{+w}{ }\PY{o}{=}\PY{+w}{ }\PY{n}{rand}\PY{p}{(}\PY{n}{n}\PY{p}{)}\PY{+w}{  }\PY{c}{\PYZsh{} Start with a random vector}
\PY{+w}{    }\PY{n}{v}\PY{+w}{ }\PY{o}{=}\PY{+w}{ }\PY{n}{v}\PY{+w}{ }\PY{o}{/}\PY{+w}{ }\PY{n}{norm}\PY{p}{(}\PY{n}{v}\PY{p}{)}\PY{+w}{  }\PY{c}{\PYZsh{} Normalize the vector}

\PY{+w}{    }\PY{c}{\PYZsh{} Iterate for a maximum number of iterations or until convergence}
\PY{+w}{    }\PY{k}{for}\PY{+w}{ }\PY{n}{\PYZus{}}\PY{+w}{ }\PY{o}{=}\PY{+w}{ }\PY{l+m+mi}{1}\PY{o}{:}\PY{n}{max\PYZus{}iter}
\PY{+w}{        }\PY{c}{\PYZsh{} Multiply the vector by the matrix A}
\PY{+w}{        }\PY{n}{v\PYZus{}new}\PY{+w}{ }\PY{o}{=}\PY{+w}{ }\PY{n}{A}\PY{+w}{ }\PY{o}{*}\PY{+w}{ }\PY{n}{v}
\PY{+w}{        }\PY{c}{\PYZsh{} Normalize the resulting vector}
\PY{+w}{        }\PY{n}{v\PYZus{}new}\PY{+w}{ }\PY{o}{=}\PY{+w}{ }\PY{n}{v\PYZus{}new}\PY{+w}{ }\PY{o}{/}\PY{+w}{ }\PY{n}{norm}\PY{p}{(}\PY{n}{v\PYZus{}new}\PY{p}{)}
\PY{+w}{        }\PY{c}{\PYZsh{} Update the vector for the next iteration}
\PY{+w}{        }\PY{n}{v}\PY{+w}{ }\PY{o}{=}\PY{+w}{ }\PY{n}{v\PYZus{}new}
\PY{+w}{    }\PY{k}{end}

\PY{+w}{    }\PY{c}{\PYZsh{} Calculate the leading eigenvalue which is done by the Rayleigh quotient: (v\PYZsq{}Av) / (v\PYZsq{}v)}
\PY{+w}{    }\PY{n}{eigenvalue}\PY{+w}{ }\PY{o}{=}\PY{+w}{ }\PY{n}{dot}\PY{p}{(}\PY{n}{A}\PY{+w}{ }\PY{o}{*}\PY{+w}{ }\PY{n}{v}\PY{p}{,}\PY{+w}{ }\PY{n}{v}\PY{p}{)}\PY{+w}{ }\PY{o}{/}\PY{+w}{ }\PY{n}{dot}\PY{p}{(}\PY{n}{v}\PY{p}{,}\PY{+w}{ }\PY{n}{v}\PY{p}{)}
\PY{+w}{    }\PY{k}{return}\PY{+w}{ }\PY{n}{eigenvalue}\PY{p}{,}\PY{+w}{ }\PY{n}{v}
\PY{k}{end}
\end{Verbatim}
\end{tcolorbox}

            \begin{tcolorbox}[breakable, size=fbox, boxrule=.5pt, pad at break*=1mm, opacityfill=0]
\prompt{Out}{outcolor}{133}{\boxspacing}
\begin{Verbatim}[commandchars=\\\{\}]
power\_method (generic function with 1 method)
\end{Verbatim}
\end{tcolorbox}
        
    \begin{tcolorbox}[breakable, size=fbox, boxrule=1pt, pad at break*=1mm,colback=cellbackground, colframe=cellborder]
\prompt{In}{incolor}{134}{\boxspacing}
\begin{Verbatim}[commandchars=\\\{\}]
\PY{c}{\PYZsh{} Function to find all eigenvalues and eigenvectors of a symmetric matrix using deflation}
\PY{k}{function}\PY{+w}{ }\PY{n}{deflation}\PY{p}{(}\PY{n}{A}\PY{p}{,}\PY{+w}{ }\PY{n}{max\PYZus{}iter}\PY{p}{)}
\PY{+w}{    }\PY{n}{k}\PY{+w}{ }\PY{o}{=}\PY{+w}{ }\PY{n}{size}\PY{p}{(}\PY{n}{A}\PY{p}{,}\PY{+w}{ }\PY{l+m+mi}{1}\PY{p}{)}
\PY{+w}{    }\PY{n}{eigenvalues}\PY{+w}{ }\PY{o}{=}\PY{+w}{ }\PY{p}{[}\PY{p}{]}
\PY{+w}{    }\PY{n}{eigenvectors}\PY{+w}{ }\PY{o}{=}\PY{+w}{ }\PY{p}{[}\PY{p}{]}

\PY{+w}{    }\PY{c}{\PYZsh{} Main loop for finding all eigenvalues and eigenvectors}
\PY{+w}{    }\PY{k}{for}\PY{+w}{ }\PY{n}{i}\PY{+w}{ }\PY{o}{=}\PY{+w}{ }\PY{l+m+mi}{1}\PY{o}{:}\PY{n}{k}
\PY{+w}{        }\PY{n}{eigenvalue}\PY{p}{,}\PY{+w}{ }\PY{n}{eigenvector}\PY{+w}{ }\PY{o}{=}\PY{+w}{ }\PY{n}{power\PYZus{}method}\PY{p}{(}\PY{n}{A}\PY{p}{,}\PY{+w}{ }\PY{n}{max\PYZus{}iter}\PY{p}{)}
\PY{+w}{        }\PY{n}{push!}\PY{p}{(}\PY{n}{eigenvalues}\PY{p}{,}\PY{+w}{ }\PY{n}{eigenvalue}\PY{p}{)}
\PY{+w}{        }\PY{n}{push!}\PY{p}{(}\PY{n}{eigenvectors}\PY{p}{,}\PY{+w}{ }\PY{n}{eigenvector}\PY{p}{)}
\PY{+w}{        }\PY{n}{A}\PY{+w}{ }\PY{o}{=}\PY{+w}{ }\PY{n}{A}\PY{+w}{ }\PY{o}{\PYZhy{}}\PY{+w}{ }\PY{n}{eigenvalue}\PY{+w}{ }\PY{o}{*}\PY{+w}{ }\PY{n}{eigenvector}\PY{+w}{ }\PY{o}{*}\PY{+w}{ }\PY{n}{transpose}\PY{p}{(}\PY{n}{eigenvector}\PY{p}{)}
\PY{+w}{    }\PY{k}{end}

\PY{+w}{    }\PY{k}{return}\PY{+w}{ }\PY{n}{eigenvalues}\PY{p}{,}\PY{+w}{ }\PY{n}{eigenvectors}
\PY{k}{end}
\end{Verbatim}
\end{tcolorbox}

            \begin{tcolorbox}[breakable, size=fbox, boxrule=.5pt, pad at break*=1mm, opacityfill=0]
\prompt{Out}{outcolor}{134}{\boxspacing}
\begin{Verbatim}[commandchars=\\\{\}]
deflation (generic function with 1 method)
\end{Verbatim}
\end{tcolorbox}
        
    \begin{tcolorbox}[breakable, size=fbox, boxrule=1pt, pad at break*=1mm,colback=cellbackground, colframe=cellborder]
\prompt{In}{incolor}{135}{\boxspacing}
\begin{Verbatim}[commandchars=\\\{\}]
\PY{c}{\PYZsh{}Function to obtain all the singular values and singular vectors}
\PY{k}{function}\PY{+w}{ }\PY{n}{singular\PYZus{}values\PYZus{}vectors}\PY{p}{(}\PY{n}{A}\PY{p}{,}\PY{+w}{ }\PY{n}{max\PYZus{}iter}\PY{p}{)}

\PY{n}{v\PYZus{}eigenvalue}\PY{p}{,}\PY{+w}{ }\PY{n}{v\PYZus{}eigenvector}\PY{+w}{ }\PY{o}{=}\PY{+w}{ }\PY{n}{power\PYZus{}method}\PY{p}{(}\PY{n}{transpose}\PY{p}{(}\PY{n}{A}\PY{p}{)}\PY{+w}{ }\PY{o}{*}\PY{+w}{ }\PY{n}{A}\PY{p}{,}\PY{+w}{ }\PY{n}{max\PYZus{}iter}\PY{p}{)}
\PY{n}{singular\PYZus{}values}\PY{p}{,}\PY{+w}{ }\PY{n}{v\PYZus{}singular\PYZus{}vectors}\PY{+w}{ }\PY{o}{=}\PY{+w}{ }\PY{n}{deflation}\PY{p}{(}\PY{n}{transpose}\PY{p}{(}\PY{n}{A}\PY{p}{)}\PY{+w}{ }\PY{o}{*}\PY{+w}{ }\PY{n}{A}\PY{p}{,}\PY{+w}{ }\PY{n}{max\PYZus{}iter}\PY{p}{)}
\PY{n}{singular\PYZus{}values}\PY{+w}{ }\PY{o}{=}\PY{+w}{ }\PY{n}{sqrt}\PY{o}{.}\PY{p}{(}\PY{n}{singular\PYZus{}values}\PY{p}{)}

\PY{n}{u\PYZus{}eigenvalue}\PY{p}{,}\PY{+w}{ }\PY{n}{u\PYZus{}eigenvector}\PY{+w}{ }\PY{o}{=}\PY{+w}{ }\PY{n}{power\PYZus{}method}\PY{p}{(}\PY{n}{A}\PY{+w}{ }\PY{o}{*}\PY{+w}{ }\PY{n}{transpose}\PY{p}{(}\PY{n}{A}\PY{p}{)}\PY{p}{,}\PY{+w}{ }\PY{n}{max\PYZus{}iter}\PY{p}{)}
\PY{n}{\PYZus{}}\PY{p}{,}\PY{+w}{ }\PY{n}{u\PYZus{}singular\PYZus{}vectors}\PY{+w}{ }\PY{o}{=}\PY{+w}{ }\PY{n}{deflation}\PY{p}{(}\PY{n}{A}\PY{+w}{ }\PY{o}{*}\PY{+w}{ }\PY{n}{transpose}\PY{p}{(}\PY{n}{A}\PY{p}{)}\PY{p}{,}\PY{+w}{ }\PY{n}{max\PYZus{}iter}\PY{p}{)}

\PY{k}{return}\PY{+w}{ }\PY{n}{singular\PYZus{}values}\PY{p}{,}\PY{+w}{ }\PY{n}{v\PYZus{}singular\PYZus{}vectors}\PY{p}{,}\PY{+w}{ }\PY{n}{u\PYZus{}singular\PYZus{}vectors}
\PY{k}{end}
\end{Verbatim}
\end{tcolorbox}

            \begin{tcolorbox}[breakable, size=fbox, boxrule=.5pt, pad at break*=1mm, opacityfill=0]
\prompt{Out}{outcolor}{135}{\boxspacing}
\begin{Verbatim}[commandchars=\\\{\}]
singular\_values\_vectors (generic function with 1 method)
\end{Verbatim}
\end{tcolorbox}
        
    \begin{tcolorbox}[breakable, size=fbox, boxrule=1pt, pad at break*=1mm,colback=cellbackground, colframe=cellborder]
\prompt{In}{incolor}{136}{\boxspacing}
\begin{Verbatim}[commandchars=\\\{\}]
\PY{c}{\PYZsh{}Function to generate a random symmetric matrix A and A\PYZca{}TA}
\PY{k}{function}\PY{+w}{ }\PY{n}{rand\PYZus{}matrix\PYZus{}generator}\PY{p}{(}\PY{n}{size}\PY{p}{,}\PY{+w}{ }\PY{n}{min\PYZus{}val}\PY{p}{,}\PY{+w}{ }\PY{n}{max\PYZus{}val}\PY{p}{)}
\PY{+w}{    }\PY{n}{X}\PY{+w}{ }\PY{o}{=}\PY{+w}{ }\PY{n}{rand}\PY{p}{(}\PY{n}{min\PYZus{}val}\PY{+w}{ }\PY{o}{:}\PY{+w}{ }\PY{n}{max\PYZus{}val}\PY{p}{,}\PY{+w}{ }\PY{n}{size}\PY{p}{,}\PY{+w}{ }\PY{n}{size}\PY{p}{)}
\PY{+w}{    }\PY{n}{A}\PY{+w}{ }\PY{o}{=}\PY{+w}{ }\PY{n}{transpose}\PY{p}{(}\PY{n}{X}\PY{p}{)}\PY{+w}{ }\PY{o}{*}\PY{+w}{ }\PY{n}{X}
\PY{+w}{    }\PY{k}{return}\PY{+w}{ }\PY{n}{A}
\PY{k}{end}
\end{Verbatim}
\end{tcolorbox}

            \begin{tcolorbox}[breakable, size=fbox, boxrule=.5pt, pad at break*=1mm, opacityfill=0]
\prompt{Out}{outcolor}{136}{\boxspacing}
\begin{Verbatim}[commandchars=\\\{\}]
rand\_matrix\_generator (generic function with 1 method)
\end{Verbatim}
\end{tcolorbox}
        
    \begin{tcolorbox}[breakable, size=fbox, boxrule=1pt, pad at break*=1mm,colback=cellbackground, colframe=cellborder]
\prompt{In}{incolor}{137}{\boxspacing}
\begin{Verbatim}[commandchars=\\\{\}]
\PY{c}{\PYZsh{}Function to compute and display all the eigenvalues and eigenvectors}
\PY{k}{function}\PY{+w}{ }\PY{n}{display\PYZus{}computations}\PY{p}{(}\PY{n}{A}\PY{p}{,}\PY{+w}{ }\PY{n}{max\PYZus{}iter\PYZus{}qr}\PY{p}{,}\PY{+w}{ }\PY{n}{max\PYZus{}iter\PYZus{}pm}\PY{p}{)}
\PY{+w}{    }\PY{c}{\PYZsh{}get leading eigenvalue from QR method}
\PY{+w}{    }\PY{n}{eigenvalue\PYZus{}qr}\PY{+w}{ }\PY{o}{=}\PY{+w}{ }\PY{n}{qr\PYZus{}method}\PY{p}{(}\PY{n}{A}\PY{p}{,}\PY{+w}{ }\PY{n}{max\PYZus{}iter\PYZus{}qr}\PY{p}{)}
\PY{+w}{    }\PY{n}{print}\PY{p}{(}\PY{l+s}{\PYZdq{}}\PY{l+s}{A\PYZsq{}s leading eigenvalue (QR method)}\PY{l+s}{\PYZdq{}}\PY{p}{)}
\PY{+w}{    }\PY{n}{display}\PY{p}{(}\PY{n}{eigenvalue\PYZus{}qr}\PY{p}{)}

\PY{+w}{    }\PY{c}{\PYZsh{}get leading eigenvector and eigenvalue from the power method}
\PY{+w}{    }\PY{n}{eigenvalue\PYZus{}pm}\PY{p}{,}\PY{+w}{ }\PY{n}{eigenvector\PYZus{}pm}\PY{+w}{ }\PY{o}{=}\PY{+w}{ }\PY{n}{power\PYZus{}method}\PY{p}{(}\PY{n}{A}\PY{p}{,}\PY{+w}{ }\PY{n}{max\PYZus{}iter\PYZus{}pm}\PY{p}{)}
\PY{+w}{    }\PY{n}{print}\PY{p}{(}\PY{l+s}{\PYZdq{}}\PY{l+s}{A\PYZsq{}s leading eigenvector (Power method):}\PY{l+s}{\PYZdq{}}\PY{p}{)}
\PY{+w}{    }\PY{n}{display}\PY{p}{(}\PY{n}{eigenvector\PYZus{}pm}\PY{p}{)}
\PY{+w}{    }\PY{n}{print}\PY{p}{(}\PY{l+s}{\PYZdq{}}\PY{l+s}{A\PYZsq{}s leading eigenvalue (Power method and Raleigh quotient):}\PY{l+s}{\PYZdq{}}\PY{p}{)}
\PY{+w}{    }\PY{n}{display}\PY{p}{(}\PY{n}{eigenvalue\PYZus{}pm}\PY{p}{)}

\PY{+w}{    }\PY{c}{\PYZsh{}get all eigenvalues and eigenvectors}
\PY{+w}{    }\PY{n}{eigenvalues}\PY{p}{,}\PY{+w}{ }\PY{n}{eigenvectors}\PY{+w}{ }\PY{o}{=}\PY{+w}{ }\PY{n}{deflation}\PY{p}{(}\PY{n}{A}\PY{p}{,}\PY{+w}{ }\PY{n}{max\PYZus{}iter\PYZus{}pm}\PY{p}{)}
\PY{+w}{    }\PY{n}{println}\PY{p}{(}\PY{l+s}{\PYZdq{}}\PY{l+s}{A\PYZsq{}s eigenvectors and corresponding eigenvalues (Deflation):}\PY{l+s}{\PYZdq{}}\PY{p}{)}
\PY{+w}{    }\PY{k}{for}\PY{+w}{ }\PY{n}{i}\PY{+w}{ }\PY{k}{in}\PY{+w}{ }\PY{n}{range}\PY{p}{(}\PY{l+m+mi}{1}\PY{p}{,}\PY{+w}{ }\PY{n}{length}\PY{p}{(}\PY{n}{eigenvectors}\PY{p}{)}\PY{p}{)}
\PY{+w}{        }\PY{n}{print}\PY{p}{(}\PY{l+s}{\PYZdq{}}\PY{l+s}{Eigenvector number }\PY{l+s}{\PYZdq{}}\PY{+w}{ }\PY{o}{*}\PY{+w}{ }\PY{n}{string}\PY{p}{(}\PY{n}{i}\PY{p}{)}\PY{+w}{ }\PY{o}{*}\PY{+w}{ }\PY{l+s}{\PYZdq{}}\PY{l+s}{ is:}\PY{l+s}{\PYZdq{}}\PY{p}{)}
\PY{+w}{        }\PY{n}{display}\PY{p}{(}\PY{n}{eigenvectors}\PY{p}{[}\PY{n}{i}\PY{p}{,}\PY{+w}{ }\PY{l+m+mi}{1}\PY{p}{]}\PY{p}{)}
\PY{+w}{        }\PY{n}{println}\PY{p}{(}\PY{l+s}{\PYZdq{}}\PY{l+s}{And has corresponding eigenvalue:}\PY{l+s}{\PYZdq{}}\PY{p}{)}
\PY{+w}{        }\PY{n}{display}\PY{p}{(}\PY{n}{eigenvalues}\PY{p}{[}\PY{n}{i}\PY{p}{,}\PY{+w}{ }\PY{l+m+mi}{1}\PY{p}{]}\PY{p}{)}
\PY{+w}{        }\PY{n}{println}\PY{p}{(}\PY{l+s}{\PYZdq{}}\PY{l+s}{\PYZdq{}}\PY{p}{)}
\PY{+w}{    }\PY{k}{end}

\PY{+w}{    }\PY{c}{\PYZsh{}get all singular vectors and singular values}
\PY{+w}{    }\PY{n}{singular\PYZus{}values}\PY{p}{,}\PY{+w}{ }\PY{n}{left\PYZus{}singular\PYZus{}vectors}\PY{p}{,}\PY{+w}{ }\PY{n}{right\PYZus{}singular\PYZus{}vectors}\PY{+w}{ }\PY{o}{=}\PY{+w}{ }\PY{n}{singular\PYZus{}values\PYZus{}vectors}\PY{p}{(}\PY{n}{A}\PY{p}{,}\PY{+w}{ }\PY{n}{max\PYZus{}iter\PYZus{}pm}\PY{p}{)}
\PY{+w}{    }\PY{n}{println}\PY{p}{(}\PY{l+s}{\PYZdq{}}\PY{l+s}{A\PYZsq{}s singular vectors for matrix V in A\PYZsq{}s SVD are:}\PY{l+s}{\PYZdq{}}\PY{p}{)}
\PY{+w}{    }\PY{k}{for}\PY{+w}{ }\PY{n}{i}\PY{+w}{ }\PY{k}{in}\PY{+w}{ }\PY{n}{range}\PY{p}{(}\PY{l+m+mi}{1}\PY{p}{,}\PY{+w}{ }\PY{n}{length}\PY{p}{(}\PY{n}{left\PYZus{}singular\PYZus{}vectors}\PY{p}{)}\PY{p}{)}
\PY{+w}{        }\PY{n}{println}\PY{p}{(}\PY{l+s}{\PYZdq{}}\PY{l+s}{Singular vector number }\PY{l+s}{\PYZdq{}}\PY{+w}{ }\PY{o}{*}\PY{+w}{ }\PY{n}{string}\PY{p}{(}\PY{n}{i}\PY{p}{)}\PY{+w}{ }\PY{o}{*}\PY{+w}{ }\PY{l+s}{\PYZdq{}}\PY{l+s}{ is:}\PY{l+s}{\PYZdq{}}\PY{p}{)}
\PY{+w}{        }\PY{n}{display}\PY{p}{(}\PY{n}{left\PYZus{}singular\PYZus{}vectors}\PY{p}{[}\PY{n}{i}\PY{p}{,}\PY{+w}{ }\PY{l+m+mi}{1}\PY{p}{]}\PY{p}{)}
\PY{+w}{    }\PY{k}{end}\PY{+w}{    }
\PY{+w}{    }\PY{n}{println}\PY{p}{(}\PY{l+s}{\PYZdq{}}\PY{l+s}{A\PYZsq{}s singular vectors for matrix U in A\PYZsq{}s SVD are:}\PY{l+s}{\PYZdq{}}\PY{p}{)}
\PY{+w}{    }\PY{k}{for}\PY{+w}{ }\PY{n}{i}\PY{+w}{ }\PY{k}{in}\PY{+w}{ }\PY{n}{range}\PY{p}{(}\PY{l+m+mi}{1}\PY{p}{,}\PY{+w}{ }\PY{n}{length}\PY{p}{(}\PY{n}{right\PYZus{}singular\PYZus{}vectors}\PY{p}{)}\PY{p}{)}
\PY{+w}{        }\PY{n}{println}\PY{p}{(}\PY{l+s}{\PYZdq{}}\PY{l+s}{Singular vector number }\PY{l+s}{\PYZdq{}}\PY{+w}{ }\PY{o}{*}\PY{+w}{ }\PY{n}{string}\PY{p}{(}\PY{n}{i}\PY{p}{)}\PY{+w}{ }\PY{o}{*}\PY{+w}{ }\PY{l+s}{\PYZdq{}}\PY{l+s}{ is:}\PY{l+s}{\PYZdq{}}\PY{p}{)}
\PY{+w}{        }\PY{n}{display}\PY{p}{(}\PY{n}{right\PYZus{}singular\PYZus{}vectors}\PY{p}{[}\PY{n}{i}\PY{p}{,}\PY{+w}{ }\PY{l+m+mi}{1}\PY{p}{]}\PY{p}{)}
\PY{+w}{    }\PY{k}{end}
\PY{+w}{    }\PY{n}{println}\PY{p}{(}\PY{l+s}{\PYZdq{}}\PY{l+s}{A\PYZsq{}s singular values are:}\PY{l+s}{\PYZdq{}}\PY{p}{)}
\PY{+w}{    }\PY{n}{display}\PY{p}{(}\PY{n}{singular\PYZus{}values}\PY{p}{)}

\PY{+w}{    }\PY{n}{println}\PY{p}{(}\PY{l+s}{\PYZdq{}}\PY{l+s}{\PYZhy{}\PYZhy{}\PYZhy{}\PYZhy{}\PYZhy{}\PYZhy{}\PYZhy{}\PYZhy{}\PYZhy{}\PYZhy{}\PYZhy{}\PYZhy{}\PYZhy{}\PYZhy{}\PYZhy{}\PYZhy{}\PYZhy{}\PYZhy{}\PYZhy{}\PYZhy{}\PYZhy{}\PYZhy{}\PYZhy{}\PYZhy{}\PYZhy{}\PYZhy{}\PYZhy{}\PYZhy{}\PYZhy{}\PYZhy{}\PYZhy{}\PYZhy{}\PYZhy{}\PYZhy{}\PYZhy{}\PYZhy{}\PYZhy{}\PYZhy{}\PYZhy{}\PYZhy{}\PYZhy{}\PYZhy{}\PYZhy{}\PYZhy{}\PYZhy{}\PYZhy{}\PYZhy{}\PYZhy{}\PYZhy{}\PYZhy{}\PYZhy{}\PYZhy{}\PYZhy{}}\PY{l+s}{\PYZdq{}}\PY{p}{)}
\PY{+w}{    }\PY{n}{println}\PY{p}{(}\PY{l+s}{\PYZdq{}}\PY{l+s}{Julia\PYZsq{}s Output to compare}\PY{l+s}{\PYZdq{}}\PY{p}{)}
\PY{+w}{    }\PY{n}{display}\PY{p}{(}\PY{n}{eigvals}\PY{p}{(}\PY{n}{A}\PY{p}{)}\PY{p}{)}
\PY{+w}{    }\PY{n}{display}\PY{p}{(}\PY{n}{eigvecs}\PY{p}{(}\PY{n}{A}\PY{p}{)}\PY{p}{)}
\PY{+w}{    }\PY{n}{display}\PY{p}{(}\PY{n}{svd}\PY{p}{(}\PY{n}{A}\PY{p}{)}\PY{p}{)}
\PY{+w}{    }\PY{k}{end}
\end{Verbatim}
\end{tcolorbox}

            \begin{tcolorbox}[breakable, size=fbox, boxrule=.5pt, pad at break*=1mm, opacityfill=0]
\prompt{Out}{outcolor}{137}{\boxspacing}
\begin{Verbatim}[commandchars=\\\{\}]
display\_computations (generic function with 1 method)
\end{Verbatim}
\end{tcolorbox}
        
    \begin{tcolorbox}[breakable, size=fbox, boxrule=1pt, pad at break*=1mm,colback=cellbackground, colframe=cellborder]
\prompt{In}{incolor}{139}{\boxspacing}
\begin{Verbatim}[commandchars=\\\{\}]
\PY{c}{\PYZsh{}Running our functions}
\PY{n}{A}\PY{+w}{ }\PY{o}{=}\PY{+w}{ }\PY{n}{rand\PYZus{}matrix\PYZus{}generator}\PY{p}{(}\PY{l+m+mi}{8}\PY{p}{,}\PY{+w}{ }\PY{l+m+mi}{1}\PY{p}{,}\PY{+w}{ }\PY{l+m+mi}{14}\PY{p}{)}
\PY{n}{display\PYZus{}computations}\PY{p}{(}\PY{n}{A}\PY{p}{,}\PY{+w}{ }\PY{l+m+mi}{11}\PY{p}{,}\PY{+w}{ }\PY{l+m+mi}{11}\PY{p}{)}
\end{Verbatim}
\end{tcolorbox}

    \begin{Verbatim}[commandchars=\\\{\}]
A's leading eigenvalue (QR method)
    \end{Verbatim}

    
    \begin{Verbatim}[commandchars=\\\{\}]
3706.6197377438066
    \end{Verbatim}

    
    \begin{Verbatim}[commandchars=\\\{\}]
A's leading eigenvector (Power method):
    \end{Verbatim}

    
    \begin{Verbatim}[commandchars=\\\{\}]
8-element Vector\{Float64\}:
 0.3513608224753594
 0.37638183990226576
 0.3237371058940461
 0.32864052652044595
 0.28750607067702877
 0.2945645790232591
 0.3102680156972951
 0.5063375358008766
    \end{Verbatim}

    
    \begin{Verbatim}[commandchars=\\\{\}]
A's leading eigenvalue (Power method and Raleigh quotient):
    \end{Verbatim}

    
    \begin{Verbatim}[commandchars=\\\{\}]
3706.6197377438125
    \end{Verbatim}

    
    \begin{Verbatim}[commandchars=\\\{\}]
A's eigenvectors and corresponding eigenvalues (Deflation):
Eigenvector number 1 is:
    \end{Verbatim}

    
    \begin{Verbatim}[commandchars=\\\{\}]
8-element Vector\{Float64\}:
 0.351360822475992
 0.3763818399045702
 0.3237371058940415
 0.32864052652215336
 0.28750607067694306
 0.2945645790215287
 0.31026801569583873
 0.5063375357995669
    \end{Verbatim}

    
    \begin{Verbatim}[commandchars=\\\{\}]
And has corresponding eigenvalue:
    \end{Verbatim}

    
    \begin{Verbatim}[commandchars=\\\{\}]
3706.619737743812
    \end{Verbatim}

    
    \begin{Verbatim}[commandchars=\\\{\}]

Eigenvector number 2 is:
    \end{Verbatim}

    
    \begin{Verbatim}[commandchars=\\\{\}]
8-element Vector\{Float64\}:
  0.14571551613931755
  0.6033476030755105
 -0.03655637427851586
  0.4279101425701151
  0.015203163364386573
 -0.4202180831804913
 -0.370047458248081
 -0.3413879093355321
    \end{Verbatim}

    
    \begin{Verbatim}[commandchars=\\\{\}]
And has corresponding eigenvalue:
    \end{Verbatim}

    
    \begin{Verbatim}[commandchars=\\\{\}]
380.82371257775554
    \end{Verbatim}

    
    \begin{Verbatim}[commandchars=\\\{\}]

Eigenvector number 3 is:
    \end{Verbatim}

    
    \begin{Verbatim}[commandchars=\\\{\}]
8-element Vector\{Float64\}:
 -0.25360418740957874
  0.27324405248271455
 -0.5867753620222746
 -0.11108035160359431
  0.6112811260590814
  0.3374063824440383
  0.010179994276021861
 -0.1294879131382918
    \end{Verbatim}

    
    \begin{Verbatim}[commandchars=\\\{\}]
And has corresponding eigenvalue:
    \end{Verbatim}

    
    \begin{Verbatim}[commandchars=\\\{\}]
316.6112888725979
    \end{Verbatim}

    
    \begin{Verbatim}[commandchars=\\\{\}]

Eigenvector number 4 is:
    \end{Verbatim}

    
    \begin{Verbatim}[commandchars=\\\{\}]
8-element Vector\{Float64\}:
 -0.14479027505085648
  0.28515556005123127
  0.38619850935107053
 -0.4310838100757833
  0.19973712688819506
 -0.30792437131169925
  0.5520123908990922
 -0.35115455981196486
    \end{Verbatim}

    
    \begin{Verbatim}[commandchars=\\\{\}]
And has corresponding eigenvalue:
    \end{Verbatim}

    
    \begin{Verbatim}[commandchars=\\\{\}]
185.83444412027958
    \end{Verbatim}

    
    \begin{Verbatim}[commandchars=\\\{\}]

Eigenvector number 5 is:
    \end{Verbatim}

    
    \begin{Verbatim}[commandchars=\\\{\}]
8-element Vector\{Float64\}:
 -0.017664417859922416
  0.4253602654085433
 -0.4078677800266019
 -0.06503699991839602
 -0.6853129067817778
  0.16333342881167237
  0.38547950712171036
  0.05696170670867869
    \end{Verbatim}

    
    \begin{Verbatim}[commandchars=\\\{\}]
And has corresponding eigenvalue:
    \end{Verbatim}

    
    \begin{Verbatim}[commandchars=\\\{\}]
74.94648539345816
    \end{Verbatim}

    
    \begin{Verbatim}[commandchars=\\\{\}]

Eigenvector number 6 is:
    \end{Verbatim}

    
    \begin{Verbatim}[commandchars=\\\{\}]
8-element Vector\{Float64\}:
 -0.5014680359460327
 -0.2388432661000973
 -0.1215065561743243
  0.6357019236098919
  0.05897685800826313
 -0.28315930677305284
  0.43081039424468226
  0.05786074408806446
    \end{Verbatim}

    
    \begin{Verbatim}[commandchars=\\\{\}]
And has corresponding eigenvalue:
    \end{Verbatim}

    
    \begin{Verbatim}[commandchars=\\\{\}]
45.27278198455453
    \end{Verbatim}

    
    \begin{Verbatim}[commandchars=\\\{\}]

Eigenvector number 7 is:
    \end{Verbatim}

    
    \begin{Verbatim}[commandchars=\\\{\}]
8-element Vector\{Float64\}:
  0.6021546427501092
 -0.3104373074456427
 -0.16400780699191936
  0.2273149350154011
  0.04986095031691271
  0.17667849492608365
  0.33207741763012194
 -0.5643500754755785
    \end{Verbatim}

    
    \begin{Verbatim}[commandchars=\\\{\}]
And has corresponding eigenvalue:
    \end{Verbatim}

    
    \begin{Verbatim}[commandchars=\\\{\}]
23.701241446165888
    \end{Verbatim}

    
    \begin{Verbatim}[commandchars=\\\{\}]

Eigenvector number 8 is:
    \end{Verbatim}

    
    \begin{Verbatim}[commandchars=\\\{\}]
8-element Vector\{Float64\}:
 -0.3942384538645349
  0.09202856509942431
  0.4361751313517652
  0.2240837200299576
 -0.1717457742789772
  0.6214834143606667
 -0.12927103460524353
 -0.4039743558842078
    \end{Verbatim}

    
    \begin{Verbatim}[commandchars=\\\{\}]
And has corresponding eigenvalue:
    \end{Verbatim}

    
    \begin{Verbatim}[commandchars=\\\{\}]
13.225754959699774
    \end{Verbatim}

    
    \begin{Verbatim}[commandchars=\\\{\}]

A's singular vectors for matrix V in A's SVD are:
Singular vector number 1 is:
    \end{Verbatim}

    
    \begin{Verbatim}[commandchars=\\\{\}]
8-element Vector\{Float64\}:
 0.35136082247487804
 0.3763818399002095
 0.3237371058941957
 0.3286405265190035
 0.2875060706769486
 0.2945645790246673
 0.31026801569854606
 0.5063375358020394
    \end{Verbatim}

    
    \begin{Verbatim}[commandchars=\\\{\}]
Singular vector number 2 is:
    \end{Verbatim}

    
    \begin{Verbatim}[commandchars=\\\{\}]
8-element Vector\{Float64\}:
 -0.13446246714199986
 -0.6145149653100952
  0.06165940230197924
 -0.42226936888861333
 -0.041741132335195885
  0.40523438581096294
  0.3687783520891321
  0.34673217042034815
    \end{Verbatim}

    
    \begin{Verbatim}[commandchars=\\\{\}]
Singular vector number 3 is:
    \end{Verbatim}

    
    \begin{Verbatim}[commandchars=\\\{\}]
8-element Vector\{Float64\}:
 -0.2618641573347747
  0.2444004526946695
 -0.5815705398649463
 -0.13592696382362113
  0.6114480151789907
  0.3562360870043977
  0.03305751175149655
 -0.11458414139462526
    \end{Verbatim}

    
    \begin{Verbatim}[commandchars=\\\{\}]

Singular vector number 4 is:
    \end{Verbatim}

    
    \begin{Verbatim}[commandchars=\\\{\}]
8-element Vector\{Float64\}:
  0.14168977911716651
 -0.2813837227396289
 -0.3935758979517547
  0.42992318919974126
 -0.19207110017692253
  0.31191827481249473
 -0.552078623662145
  0.3493367238365371
    \end{Verbatim}

    
    \begin{Verbatim}[commandchars=\\\{\}]
Singular vector number 5 is:
    \end{Verbatim}

    
    \begin{Verbatim}[commandchars=\\\{\}]
8-element Vector\{Float64\}:
  0.016279171892507566
 -0.42603601228306925
  0.4075128248518765
  0.06682537331644675
  0.6854630454405086
 -0.1641100949185373
 -0.38428981453502836
 -0.05679436933958514
    \end{Verbatim}

    
    
    \begin{Verbatim}[commandchars=\\\{\}]
8-element Vector\{Float64\}:
  0.5015891838415605
  0.23685729988522544
  0.1233790243566879
 -0.6353841632853213
 -0.05580897256448246
  0.28241750911344493
 -0.43256709496941037
 -0.05816278033605849
    \end{Verbatim}

    
    \begin{Verbatim}[commandchars=\\\{\}]
Singular vector number 6 is:
Singular vector number 7 is:
    \end{Verbatim}

    
    \begin{Verbatim}[commandchars=\\\{\}]
8-element Vector\{Float64\}:
  0.6022206153255303
 -0.31048401690461275
 -0.16418563883669587
  0.22732074756374882
  0.0499012843025556
  0.17643798761033988
  0.3321917950922549
 -0.5642042795627437
    \end{Verbatim}

    
    \begin{Verbatim}[commandchars=\\\{\}]
Singular vector number 8 is:
    \end{Verbatim}

    
    \begin{Verbatim}[commandchars=\\\{\}]
8-element Vector\{Float64\}:
 -0.3938768218942008
  0.09183966117273942
  0.4360753132162502
  0.22422435716256298
 -0.17171535869363252
  0.621588832271
 -0.12906804033445296
 -0.4043153284473653
    \end{Verbatim}

    
    \begin{Verbatim}[commandchars=\\\{\}]
A's singular vectors for matrix U in A's SVD are:
Singular vector number 1 is:
    \end{Verbatim}

    
    \begin{Verbatim}[commandchars=\\\{\}]
8-element Vector\{Float64\}:
 0.35136082247487804
 0.3763818399002095
 0.3237371058941957
 0.3286405265190035
 0.2875060706769486
 0.2945645790246673
 0.31026801569854606
 0.5063375358020394
    \end{Verbatim}

    
    \begin{Verbatim}[commandchars=\\\{\}]
Singular vector number 2 is:
    \end{Verbatim}

    
    \begin{Verbatim}[commandchars=\\\{\}]
8-element Vector\{Float64\}:
  0.10775382983545327
  0.6356244759336097
 -0.11911648826083002
  0.40658656978844104
  0.10226985221743627
 -0.3677539301662794
 -0.36359088384543137
 -0.3563268928963543
    \end{Verbatim}

    
    \begin{Verbatim}[commandchars=\\\{\}]
Singular vector number 3 is:
    \end{Verbatim}

    
    \begin{Verbatim}[commandchars=\\\{\}]
8-element Vector\{Float64\}:
 -0.27853379311488524
  0.1540105651328689
 -0.5668352883785468
 -0.1951390309809521
  0.5993972654035542
  0.41063938223714014
  0.08566497714236794
 -0.06385854704698624
    \end{Verbatim}

    
    \begin{Verbatim}[commandchars=\\\{\}]
Singular vector number 4 is:
    \end{Verbatim}

    
    \begin{Verbatim}[commandchars=\\\{\}]
8-element Vector\{Float64\}:
 -0.14150897886228334
  0.28128021660609265
  0.39394552047274967
 -0.4297981826037624
  0.19168033791736275
 -0.3121837417520839
  0.5520246645425246
 -0.34929332509455097
    \end{Verbatim}

    
    \begin{Verbatim}[commandchars=\\\{\}]
Singular vector number 5 is:
    \end{Verbatim}

    
    \begin{Verbatim}[commandchars=\\\{\}]
8-element Vector\{Float64\}:
 -0.016262189912691027
  0.42604403980048755
 -0.40750863703018686
 -0.06684690003569217
 -0.6854649303289657
  0.1641196507795538
  0.384275179863193
  0.056792391122013605
    \end{Verbatim}

    
    \begin{Verbatim}[commandchars=\\\{\}]
Singular vector number 6 is:
    \end{Verbatim}

    
    \begin{Verbatim}[commandchars=\\\{\}]
8-element Vector\{Float64\}:
  0.5015907494037682
  0.23681773135439324
  0.12341682697445912
 -0.635377936899505
 -0.0557453538219592
  0.2824022944299186
 -0.43260272491583596
 -0.05816810443005255
    \end{Verbatim}

    
    \begin{Verbatim}[commandchars=\\\{\}]
Singular vector number 7 is:
    \end{Verbatim}

    
    \begin{Verbatim}[commandchars=\\\{\}]
8-element Vector\{Float64\}:
  0.6022191959178412
 -0.3104829537785075
 -0.16418434972822868
  0.22732225992855612
  0.04990087109688192
  0.17643959910188997
  0.3321910331416273
 -0.5642061266586615
    \end{Verbatim}

    
    \begin{Verbatim}[commandchars=\\\{\}]
Singular vector number 8 is:
    \end{Verbatim}

    
    \begin{Verbatim}[commandchars=\\\{\}]
8-element Vector\{Float64\}:
 -0.3405179253335854
  0.29767210159497315
  0.3649656831408873
  0.3426645868690311
 -0.12033291313833257
  0.4720313360557701
 -0.24017315312850274
 -0.4998431332000915
    \end{Verbatim}

    
    \begin{Verbatim}[commandchars=\\\{\}]
A's singular values are:
    \end{Verbatim}

    
    \begin{Verbatim}[commandchars=\\\{\}]
8-element Vector\{Float64\}:
 3706.619737743812
  380.9711771879006
  316.44171865432446
  185.82225295647052
   74.94671524695414
   45.272401503073965
   23.701242414976036
   13.225752844573659
    \end{Verbatim}

    
    \begin{Verbatim}[commandchars=\\\{\}]
-----------------------------------------------------
Julia's Output to compare
    \end{Verbatim}

    
    \begin{Verbatim}[commandchars=\\\{\}]
8-element Vector\{Float64\}:
   13.225752844526445
   23.70124241494957
   45.272401501636026
   74.94671524727146
  185.8222528779295
  316.4393799764263
  380.97251739344813
 3706.619737743812
    \end{Verbatim}

    
    
    \begin{Verbatim}[commandchars=\\\{\}]
8×8 Matrix\{Float64\}:
  0.393879   -0.602221   0.501589   …   0.262792    0.133212   -0.351361
 -0.0918407   0.310483   0.236862      -0.240159    0.615666   -0.376382
 -0.436076    0.164185   0.123375       0.581135   -0.0644282  -0.323737
 -0.224224   -0.22732   -0.635385       0.138846    0.421613   -0.328641
  0.171716   -0.049901  -0.0558159     -0.611157    0.0446528  -0.287506
 -0.621588   -0.176439   0.282419   …  -0.359027   -0.403529   -0.294565
  0.129069   -0.332191  -0.432563      -0.0356102  -0.368613   -0.310268
  0.404314    0.564205  -0.0581619      0.112194   -0.347271   -0.506338
    \end{Verbatim}

    
    
    \begin{Verbatim}[commandchars=\\\{\}]
SVD\{Float64, Float64, Matrix\{Float64\}, Vector\{Float64\}\}
U factor:
8×8 Matrix\{Float64\}:
 -0.351361   0.133212    0.262792   …   0.501589   -0.602221   0.393879
 -0.376382   0.615666   -0.240159       0.236862    0.310483  -0.0918407
 -0.323737  -0.0644282   0.581135       0.123375    0.164185  -0.436076
 -0.328641   0.421613    0.138846      -0.635385   -0.22732   -0.224224
 -0.287506   0.0446528  -0.611157      -0.0558159  -0.049901   0.171716
 -0.294565  -0.403529   -0.359027   …   0.282419   -0.176439  -0.621588
 -0.310268  -0.368613   -0.0356102     -0.432563   -0.332191   0.129069
 -0.506338  -0.347271    0.112194      -0.0581619   0.564205   0.404314
singular values:
8-element Vector\{Float64\}:
 3706.6197377438125
  380.9725173934479
  316.4393799764264
  185.8222528779295
   74.9467152472715
   45.27240150163626
   23.70124241494977
   13.225752844526577
Vt factor:
8×8 Matrix\{Float64\}:
 -0.351361  -0.376382   -0.323737   …  -0.294565  -0.310268   -0.506338
  0.133212   0.615666   -0.0644282     -0.403529  -0.368613   -0.347271
  0.262792  -0.240159    0.581135      -0.359027  -0.0356102   0.112194
 -0.14168    0.281375    0.393597      -0.311931   0.552078   -0.349332
 -0.016281   0.426035   -0.407513       0.164109   0.384291    0.0567946
  0.501589   0.236862    0.123375   …   0.282419  -0.432563   -0.0581619
 -0.602221   0.310483    0.164185      -0.176439  -0.332191    0.564205
  0.393879  -0.0918407  -0.436076      -0.621588   0.129069    0.404314
    \end{Verbatim}

    
    \begin{tcolorbox}[breakable, size=fbox, boxrule=1pt, pad at break*=1mm,colback=cellbackground, colframe=cellborder]
\prompt{In}{incolor}{140}{\boxspacing}
\begin{Verbatim}[commandchars=\\\{\}]
\PY{n}{A\PYZus{}QR}\PY{+w}{ }\PY{o}{=}\PY{+w}{ }\PY{p}{[}\PY{p}{[}\PY{l+m+mi}{7}\PY{+w}{ }\PY{l+m+mi}{1}\PY{p}{]}\PY{p}{;}\PY{+w}{ }\PY{p}{[}\PY{l+m+mi}{1}\PY{+w}{ }\PY{l+m+mi}{7}\PY{p}{]}\PY{p}{]}
\end{Verbatim}
\end{tcolorbox}

            \begin{tcolorbox}[breakable, size=fbox, boxrule=.5pt, pad at break*=1mm, opacityfill=0]
\prompt{Out}{outcolor}{140}{\boxspacing}
\begin{Verbatim}[commandchars=\\\{\}]
2×2 Matrix\{Int64\}:
 7  1
 1  7
\end{Verbatim}
\end{tcolorbox}
        
    \begin{tcolorbox}[breakable, size=fbox, boxrule=1pt, pad at break*=1mm,colback=cellbackground, colframe=cellborder]
\prompt{In}{incolor}{141}{\boxspacing}
\begin{Verbatim}[commandchars=\\\{\}]
\PY{c}{\PYZsh{} Function to approximate the leading eigenvalue of a matrix using the QR method (example)}
\PY{k}{function}\PY{+w}{ }\PY{n}{qr\PYZus{}method\PYZus{}example}\PY{p}{(}\PY{n}{A}\PY{p}{,}\PY{+w}{ }\PY{n}{max\PYZus{}iter}\PY{p}{)}
\PY{+w}{    }\PY{c}{\PYZsh{} Initialize A\PYZus{}k with the input matrix A}
\PY{+w}{    }\PY{n}{A\PYZus{}k}\PY{+w}{ }\PY{o}{=}\PY{+w}{ }\PY{n}{copy}\PY{p}{(}\PY{n}{A}\PY{p}{)}

\PY{+w}{    }\PY{c}{\PYZsh{} Iterate up to a maximum number of iterations}
\PY{+w}{    }\PY{k}{for}\PY{+w}{ }\PY{n}{k}\PY{+w}{ }\PY{o}{=}\PY{+w}{ }\PY{l+m+mi}{1}\PY{o}{:}\PY{n}{max\PYZus{}iter}
\PY{+w}{        }\PY{c}{\PYZsh{} Perform QR decomposition on A\PYZus{}k}
\PY{+w}{        }\PY{n}{Q}\PY{p}{,}\PY{+w}{ }\PY{n}{R}\PY{+w}{ }\PY{o}{=}\PY{+w}{ }\PY{n}{qr}\PY{p}{(}\PY{n}{A\PYZus{}k}\PY{p}{)}
\PY{+w}{        }\PY{n}{print}\PY{p}{(}\PY{l+s}{\PYZdq{}}\PY{l+s}{Q: }\PY{l+s}{\PYZdq{}}\PY{p}{)}
\PY{+w}{        }\PY{n}{display}\PY{p}{(}\PY{n}{Q}\PY{p}{)}
\PY{+w}{        }\PY{n}{print}\PY{p}{(}\PY{l+s}{\PYZdq{}}\PY{l+s}{R: }\PY{l+s}{\PYZdq{}}\PY{p}{)}
\PY{+w}{        }\PY{n}{display}\PY{p}{(}\PY{n}{R}\PY{p}{)}
\PY{+w}{        }\PY{c}{\PYZsh{} Form the next matrix A\PYZus{}k by multiplying R and Q}
\PY{+w}{        }\PY{n}{A\PYZus{}k}\PY{+w}{ }\PY{o}{=}\PY{+w}{ }\PY{n}{R}\PY{+w}{ }\PY{o}{*}\PY{+w}{ }\PY{n}{Q}
\PY{+w}{    }\PY{k}{end}

\PY{+w}{    }\PY{n}{print}\PY{p}{(}\PY{l+s}{\PYZdq{}}\PY{l+s}{The leading eigevalue of A is: }\PY{l+s}{\PYZdq{}}\PY{p}{)}
\PY{+w}{    }\PY{c}{\PYZsh{} \PYZsh{} Return the leading eigenvalue, which is the first diagonal element of A\PYZus{}k}
\PY{+w}{    }\PY{n}{display}\PY{p}{(}\PY{n}{A\PYZus{}k}\PY{p}{[}\PY{l+m+mi}{1}\PY{p}{,}\PY{l+m+mi}{1}\PY{p}{]}\PY{p}{)}
\PY{+w}{    }\PY{k}{return}\PY{+w}{ }\PY{n}{A\PYZus{}k}\PY{p}{[}\PY{l+m+mi}{1}\PY{p}{,}\PY{l+m+mi}{1}\PY{p}{]}
\PY{k}{end}
\end{Verbatim}
\end{tcolorbox}

            \begin{tcolorbox}[breakable, size=fbox, boxrule=.5pt, pad at break*=1mm, opacityfill=0]
\prompt{Out}{outcolor}{141}{\boxspacing}
\begin{Verbatim}[commandchars=\\\{\}]
qr\_method\_example (generic function with 1 method)
\end{Verbatim}
\end{tcolorbox}
        
    \begin{tcolorbox}[breakable, size=fbox, boxrule=1pt, pad at break*=1mm,colback=cellbackground, colframe=cellborder]
\prompt{In}{incolor}{142}{\boxspacing}
\begin{Verbatim}[commandchars=\\\{\}]
\PY{n}{qr\PYZus{}method\PYZus{}example}\PY{p}{(}\PY{n}{A\PYZus{}QR}\PY{p}{,}\PY{+w}{ }\PY{l+m+mi}{5}\PY{p}{)}
\end{Verbatim}
\end{tcolorbox}

    \begin{Verbatim}[commandchars=\\\{\}]
Q:
    \end{Verbatim}

    
    \begin{Verbatim}[commandchars=\\\{\}]
2×2 LinearAlgebra.QRCompactWYQ\{Float64, Matrix\{Float64\}, Matrix\{Float64\}\}:
 -0.989949  -0.141421
 -0.141421   0.989949
    \end{Verbatim}

    
    \begin{Verbatim}[commandchars=\\\{\}]
R:
    \end{Verbatim}

    
    \begin{Verbatim}[commandchars=\\\{\}]
2×2 Matrix\{Float64\}:
 -7.07107  -1.9799
  0.0       6.78823
    \end{Verbatim}

    
    \begin{Verbatim}[commandchars=\\\{\}]
Q:
    \end{Verbatim}

    
    \begin{Verbatim}[commandchars=\\\{\}]
2×2 LinearAlgebra.QRCompactWYQ\{Float64, Matrix\{Float64\}, Matrix\{Float64\}\}:
 -0.991417  0.130736
  0.130736  0.991417
    \end{Verbatim}

    
    \begin{Verbatim}[commandchars=\\\{\}]
R:
    \end{Verbatim}

    
    \begin{Verbatim}[commandchars=\\\{\}]
2×2 Matrix\{Float64\}:
 -7.34302  1.83031
  0.0      6.53682
    \end{Verbatim}

    
    \begin{Verbatim}[commandchars=\\\{\}]
Q:
    \end{Verbatim}

    
    \begin{Verbatim}[commandchars=\\\{\}]
2×2 LinearAlgebra.QRCompactWYQ\{Float64, Matrix\{Float64\}, Matrix\{Float64\}\}:
 -0.993603  -0.112927
 -0.112927   0.993603
    \end{Verbatim}

    
    \begin{Verbatim}[commandchars=\\\{\}]
R:
    \end{Verbatim}

    
    \begin{Verbatim}[commandchars=\\\{\}]
2×2 Matrix\{Float64\}:
 -7.5677  -1.58098
  0.0      6.34275
    \end{Verbatim}

    
    \begin{Verbatim}[commandchars=\\\{\}]
Q:
    \end{Verbatim}

    
    \begin{Verbatim}[commandchars=\\\{\}]
2×2 LinearAlgebra.QRCompactWYQ\{Float64, Matrix\{Float64\}, Matrix\{Float64\}\}:
 -0.995699   0.0926481
  0.0926481  0.995699
    \end{Verbatim}

    
    \begin{Verbatim}[commandchars=\\\{\}]
R:
    \end{Verbatim}

    
    \begin{Verbatim}[commandchars=\\\{\}]
2×2 Matrix\{Float64\}:
 -7.73108  1.29707
  0.0      6.20871
    \end{Verbatim}

    
    \begin{Verbatim}[commandchars=\\\{\}]
Q:
    \end{Verbatim}

    
    \begin{Verbatim}[commandchars=\\\{\}]
2×2 LinearAlgebra.QRCompactWYQ\{Float64, Matrix\{Float64\}, Matrix\{Float64\}\}:
 -0.997304   -0.0733787
 -0.0733787   0.997304
    \end{Verbatim}

    
    \begin{Verbatim}[commandchars=\\\{\}]
R:
    \end{Verbatim}

    
    \begin{Verbatim}[commandchars=\\\{\}]
2×2 Matrix\{Float64\}:
 -7.83913  -1.0273
  0.0       6.12313
    \end{Verbatim}

    
    \begin{Verbatim}[commandchars=\\\{\}]
The leading eigevalue of A is:
    \end{Verbatim}

    
    \begin{Verbatim}[commandchars=\\\{\}]
7.893377271188352
    \end{Verbatim}

    
            \begin{tcolorbox}[breakable, size=fbox, boxrule=.5pt, pad at break*=1mm, opacityfill=0]
\prompt{Out}{outcolor}{142}{\boxspacing}
\begin{Verbatim}[commandchars=\\\{\}]
7.893377271188352
\end{Verbatim}
\end{tcolorbox}
        
    \begin{tcolorbox}[breakable, size=fbox, boxrule=1pt, pad at break*=1mm,colback=cellbackground, colframe=cellborder]
\prompt{In}{incolor}{143}{\boxspacing}
\begin{Verbatim}[commandchars=\\\{\}]
\PY{n}{A\PYZus{}Power}\PY{+w}{ }\PY{o}{=}\PY{+w}{ }\PY{p}{[}\PY{p}{[}\PY{l+m+mi}{7}\PY{+w}{ }\PY{l+m+mi}{3}\PY{+w}{ }\PY{l+m+mi}{5}\PY{p}{]}\PY{p}{;}\PY{+w}{ }\PY{p}{[}\PY{l+m+mi}{3}\PY{+w}{ }\PY{l+m+mi}{5}\PY{+w}{ }\PY{l+m+mi}{1}\PY{p}{]}\PY{p}{;}\PY{+w}{ }\PY{p}{[}\PY{l+m+mi}{5}\PY{+w}{ }\PY{l+m+mi}{1}\PY{+w}{ }\PY{l+m+mi}{1}\PY{p}{]}\PY{p}{]}
\end{Verbatim}
\end{tcolorbox}

            \begin{tcolorbox}[breakable, size=fbox, boxrule=.5pt, pad at break*=1mm, opacityfill=0]
\prompt{Out}{outcolor}{143}{\boxspacing}
\begin{Verbatim}[commandchars=\\\{\}]
3×3 Matrix\{Int64\}:
 7  3  5
 3  5  1
 5  1  1
\end{Verbatim}
\end{tcolorbox}
        
    \begin{tcolorbox}[breakable, size=fbox, boxrule=1pt, pad at break*=1mm,colback=cellbackground, colframe=cellborder]
\prompt{In}{incolor}{144}{\boxspacing}
\begin{Verbatim}[commandchars=\\\{\}]
\PY{c}{\PYZsh{} Define a function to compute the leading eigenvalue using the Power Method (Example)}
\PY{k}{function}\PY{+w}{ }\PY{n}{power\PYZus{}method\PYZus{}example}\PY{p}{(}\PY{n}{A}\PY{p}{,}\PY{+w}{ }\PY{n}{max\PYZus{}iter}\PY{p}{)}
\PY{+w}{    }\PY{c}{\PYZsh{} Determine the size of the matrix A and initialize a random vector of this size }
\PY{+w}{    }\PY{n}{n}\PY{p}{,}\PY{+w}{ }\PY{n}{\PYZus{}}\PY{+w}{ }\PY{o}{=}\PY{+w}{ }\PY{n}{size}\PY{p}{(}\PY{n}{A}\PY{p}{)}
\PY{+w}{    }\PY{n}{v}\PY{+w}{ }\PY{o}{=}\PY{+w}{ }\PY{n}{rand}\PY{p}{(}\PY{n}{n}\PY{p}{)}\PY{+w}{  }\PY{c}{\PYZsh{} Start with a random vector}
\PY{+w}{    }\PY{n}{v}\PY{+w}{ }\PY{o}{=}\PY{+w}{ }\PY{n}{v}\PY{+w}{ }\PY{o}{/}\PY{+w}{ }\PY{n}{norm}\PY{p}{(}\PY{n}{v}\PY{p}{)}\PY{+w}{  }\PY{c}{\PYZsh{} Normalize the vector}

\PY{+w}{    }\PY{c}{\PYZsh{} Iterate for a maximum number of iterations or until convergence}
\PY{+w}{    }\PY{k}{for}\PY{+w}{ }\PY{n}{\PYZus{}}\PY{+w}{ }\PY{o}{=}\PY{+w}{ }\PY{l+m+mi}{1}\PY{o}{:}\PY{n}{max\PYZus{}iter}
\PY{+w}{        }\PY{c}{\PYZsh{} Multiply the vector by the matrix A}
\PY{+w}{        }\PY{n}{v\PYZus{}new}\PY{+w}{ }\PY{o}{=}\PY{+w}{ }\PY{n}{A}\PY{+w}{ }\PY{o}{*}\PY{+w}{ }\PY{n}{v}\PY{+w}{        }
\PY{+w}{        }\PY{n}{print}\PY{p}{(}\PY{l+s}{\PYZdq{}}\PY{l+s}{The updated vector is }\PY{l+s}{\PYZdq{}}\PY{p}{)}
\PY{+w}{        }\PY{n}{display}\PY{p}{(}\PY{n}{v\PYZus{}new}\PY{p}{)}
\PY{+w}{        }\PY{c}{\PYZsh{} Normalize the resulting vector}
\PY{+w}{        }\PY{n}{v\PYZus{}new}\PY{+w}{ }\PY{o}{=}\PY{+w}{ }\PY{n}{v\PYZus{}new}\PY{+w}{ }\PY{o}{/}\PY{+w}{ }\PY{n}{norm}\PY{p}{(}\PY{n}{v\PYZus{}new}\PY{p}{)}
\PY{+w}{        }\PY{c}{\PYZsh{} Update the vector for the next iteration}
\PY{+w}{        }\PY{n}{v}\PY{+w}{ }\PY{o}{=}\PY{+w}{ }\PY{n}{v\PYZus{}new}
\PY{+w}{    }\PY{k}{end}

\PY{+w}{    }\PY{c}{\PYZsh{} Calculate the leading eigenvalue which is done by the Rayleigh quotient: (v\PYZsq{}Av) / (v\PYZsq{}v)}
\PY{+w}{    }\PY{n}{eigenvalue}\PY{+w}{ }\PY{o}{=}\PY{+w}{ }\PY{n}{dot}\PY{p}{(}\PY{n}{A}\PY{+w}{ }\PY{o}{*}\PY{+w}{ }\PY{n}{v}\PY{p}{,}\PY{+w}{ }\PY{n}{v}\PY{p}{)}\PY{+w}{ }\PY{o}{/}\PY{+w}{ }\PY{n}{dot}\PY{p}{(}\PY{n}{v}\PY{p}{,}\PY{+w}{ }\PY{n}{v}\PY{p}{)}
\PY{+w}{    }\PY{n}{print}\PY{p}{(}\PY{l+s}{\PYZdq{}}\PY{l+s}{The leading eigenvalue is }\PY{l+s}{\PYZdq{}}\PY{p}{)}
\PY{+w}{    }\PY{n}{display}\PY{p}{(}\PY{n}{eigenvalue}\PY{p}{)}
\PY{+w}{    }\PY{n}{print}\PY{p}{(}\PY{l+s}{\PYZdq{}}\PY{l+s}{The leading eigenvector is }\PY{l+s}{\PYZdq{}}\PY{p}{)}
\PY{+w}{    }\PY{n}{display}\PY{p}{(}\PY{n}{v}\PY{p}{)}
\PY{+w}{    }\PY{k}{return}\PY{+w}{ }\PY{n}{eigenvalue}\PY{p}{,}\PY{+w}{ }\PY{n}{v}
\PY{k}{end}
\end{Verbatim}
\end{tcolorbox}

            \begin{tcolorbox}[breakable, size=fbox, boxrule=.5pt, pad at break*=1mm, opacityfill=0]
\prompt{Out}{outcolor}{144}{\boxspacing}
\begin{Verbatim}[commandchars=\\\{\}]
power\_method\_example (generic function with 1 method)
\end{Verbatim}
\end{tcolorbox}
        
    \begin{tcolorbox}[breakable, size=fbox, boxrule=1pt, pad at break*=1mm,colback=cellbackground, colframe=cellborder]
\prompt{In}{incolor}{153}{\boxspacing}
\begin{Verbatim}[commandchars=\\\{\}]
\PY{n}{power\PYZus{}method\PYZus{}example}\PY{p}{(}\PY{n}{A\PYZus{}Power}\PY{p}{,}\PY{+w}{ }\PY{l+m+mi}{5}\PY{p}{)}
\end{Verbatim}
\end{tcolorbox}

    \begin{Verbatim}[commandchars=\\\{\}]
The updated vector is
    \end{Verbatim}

    
    \begin{Verbatim}[commandchars=\\\{\}]
3-element Vector\{Float64\}:
 8.161030539868374
 4.627094182971236
 3.3134414620023938
    \end{Verbatim}

    
    \begin{Verbatim}[commandchars=\\\{\}]
The updated vector is
    \end{Verbatim}

    
    \begin{Verbatim}[commandchars=\\\{\}]
3-element Vector\{Float64\}:
 8.802075512800766
 5.119083565097348
 4.899340934828883
    \end{Verbatim}

    
    \begin{Verbatim}[commandchars=\\\{\}]
The updated vector is
    \end{Verbatim}

    
    \begin{Verbatim}[commandchars=\\\{\}]
3-element Vector\{Float64\}:
 8.979684450238407
 5.035582211875657
 4.7814018188575576
    \end{Verbatim}

    
    \begin{Verbatim}[commandchars=\\\{\}]
The updated vector is
    \end{Verbatim}

    
    \begin{Verbatim}[commandchars=\\\{\}]
3-element Vector\{Float64\}:
 8.974382615152209
 5.012466428032355
 4.820158385228858
    \end{Verbatim}

    
    \begin{Verbatim}[commandchars=\\\{\}]
The updated vector is
    \end{Verbatim}

    
    \begin{Verbatim}[commandchars=\\\{\}]
3-element Vector\{Float64\}:
 8.980518953209506
 5.0034304465975765
 4.818365892732459
    \end{Verbatim}

    
    \begin{Verbatim}[commandchars=\\\{\}]
The leading eigenvalue is
    \end{Verbatim}

    
    \begin{Verbatim}[commandchars=\\\{\}]
11.35345127702423
    \end{Verbatim}

    
    \begin{Verbatim}[commandchars=\\\{\}]
The leading eigenvector is
    \end{Verbatim}

    
    \begin{Verbatim}[commandchars=\\\{\}]
3-element Vector\{Float64\}:
 0.7909951454764633
 0.4406971595526109
 0.42439685837071095
    \end{Verbatim}

    
            \begin{tcolorbox}[breakable, size=fbox, boxrule=.5pt, pad at break*=1mm, opacityfill=0]
\prompt{Out}{outcolor}{153}{\boxspacing}
\begin{Verbatim}[commandchars=\\\{\}]
(11.35345127702423, [0.7909951454764633, 0.4406971595526109,
0.42439685837071095])
\end{Verbatim}
\end{tcolorbox}
        
    \begin{tcolorbox}[breakable, size=fbox, boxrule=1pt, pad at break*=1mm,colback=cellbackground, colframe=cellborder]
\prompt{In}{incolor}{145}{\boxspacing}
\begin{Verbatim}[commandchars=\\\{\}]
\PY{n}{A\PYZus{}Deflation}\PY{+w}{ }\PY{o}{=}\PY{+w}{ }\PY{p}{[}\PY{p}{[}\PY{l+m+mi}{7}\PY{+w}{ }\PY{l+m+mi}{3}\PY{+w}{ }\PY{l+m+mi}{5}\PY{p}{]}\PY{p}{;}\PY{+w}{ }\PY{p}{[}\PY{l+m+mi}{3}\PY{+w}{ }\PY{l+m+mi}{5}\PY{+w}{ }\PY{l+m+mi}{1}\PY{p}{]}\PY{p}{;}\PY{+w}{ }\PY{p}{[}\PY{l+m+mi}{5}\PY{+w}{ }\PY{l+m+mi}{1}\PY{+w}{ }\PY{l+m+mi}{1}\PY{p}{]}\PY{p}{]}
\end{Verbatim}
\end{tcolorbox}

            \begin{tcolorbox}[breakable, size=fbox, boxrule=.5pt, pad at break*=1mm, opacityfill=0]
\prompt{Out}{outcolor}{145}{\boxspacing}
\begin{Verbatim}[commandchars=\\\{\}]
3×3 Matrix\{Int64\}:
 7  3  5
 3  5  1
 5  1  1
\end{Verbatim}
\end{tcolorbox}
        
    \begin{tcolorbox}[breakable, size=fbox, boxrule=1pt, pad at break*=1mm,colback=cellbackground, colframe=cellborder]
\prompt{In}{incolor}{156}{\boxspacing}
\begin{Verbatim}[commandchars=\\\{\}]
\PY{c}{\PYZsh{} Function to find all eigenvalues and eigenvectors of a symmetric matrix using deflation (Example)}
\PY{k}{function}\PY{+w}{ }\PY{n}{deflation\PYZus{}example}\PY{p}{(}\PY{n}{A}\PY{p}{,}\PY{+w}{ }\PY{n}{max\PYZus{}iter}\PY{p}{)}
\PY{+w}{    }\PY{n}{k}\PY{+w}{ }\PY{o}{=}\PY{+w}{ }\PY{n}{size}\PY{p}{(}\PY{n}{A}\PY{p}{,}\PY{+w}{ }\PY{l+m+mi}{1}\PY{p}{)}
\PY{+w}{    }\PY{n}{eigenvalues}\PY{+w}{ }\PY{o}{=}\PY{+w}{ }\PY{p}{[}\PY{p}{]}
\PY{+w}{    }\PY{n}{eigenvectors}\PY{+w}{ }\PY{o}{=}\PY{+w}{ }\PY{p}{[}\PY{p}{]}

\PY{+w}{    }\PY{c}{\PYZsh{} Main loop for finding all eigenvalues and eigenvectors}
\PY{+w}{    }\PY{k}{for}\PY{+w}{ }\PY{n}{i}\PY{+w}{ }\PY{o}{=}\PY{+w}{ }\PY{l+m+mi}{1}\PY{o}{:}\PY{n}{k}
\PY{+w}{        }\PY{n}{eigenvalue}\PY{p}{,}\PY{+w}{ }\PY{n}{eigenvector}\PY{+w}{ }\PY{o}{=}\PY{+w}{ }\PY{n}{power\PYZus{}method}\PY{p}{(}\PY{n}{A}\PY{p}{,}\PY{+w}{ }\PY{n}{max\PYZus{}iter}\PY{p}{)}
\PY{+w}{        }\PY{n}{push!}\PY{p}{(}\PY{n}{eigenvalues}\PY{p}{,}\PY{+w}{ }\PY{n}{eigenvalue}\PY{p}{)}
\PY{+w}{        }\PY{n}{push!}\PY{p}{(}\PY{n}{eigenvectors}\PY{p}{,}\PY{+w}{ }\PY{n}{eigenvector}\PY{p}{)}
\PY{+w}{        }\PY{n}{A}\PY{+w}{ }\PY{o}{=}\PY{+w}{ }\PY{n}{A}\PY{+w}{ }\PY{o}{\PYZhy{}}\PY{+w}{ }\PY{n}{eigenvalue}\PY{+w}{ }\PY{o}{*}\PY{+w}{ }\PY{n}{eigenvector}\PY{+w}{ }\PY{o}{*}\PY{+w}{ }\PY{n}{transpose}\PY{p}{(}\PY{n}{eigenvector}\PY{p}{)}
\PY{+w}{        }\PY{n}{print}\PY{p}{(}\PY{l+s}{\PYZdq{}}\PY{l+s}{The new matrix is }\PY{l+s}{\PYZdq{}}\PY{p}{)}
\PY{+w}{        }\PY{n}{display}\PY{p}{(}\PY{n}{A}\PY{p}{)}
\PY{+w}{    }\PY{k}{end}

\PY{+w}{    }\PY{n}{print}\PY{p}{(}\PY{l+s}{\PYZdq{}}\PY{l+s}{The list of eigenvalues is}\PY{l+s}{\PYZdq{}}\PY{p}{)}
\PY{+w}{    }\PY{k}{for}\PY{+w}{ }\PY{n}{i}\PY{+w}{ }\PY{o}{=}\PY{+w}{ }\PY{l+m+mi}{1}\PY{o}{:}\PY{n}{size}\PY{p}{(}\PY{n}{eigenvalues}\PY{p}{,}\PY{+w}{ }\PY{l+m+mi}{1}\PY{p}{)}
\PY{+w}{        }\PY{n}{display}\PY{p}{(}\PY{n}{eigenvalues}\PY{p}{[}\PY{n}{i}\PY{p}{]}\PY{p}{)}
\PY{+w}{    }\PY{k}{end}
\PY{+w}{    }\PY{n}{print}\PY{p}{(}\PY{l+s}{\PYZdq{}}\PY{l+s}{The list of eigenvectors is}\PY{l+s}{\PYZdq{}}\PY{p}{)}
\PY{+w}{    }\PY{k}{for}\PY{+w}{ }\PY{n}{i}\PY{+w}{ }\PY{o}{=}\PY{+w}{ }\PY{l+m+mi}{1}\PY{o}{:}\PY{n}{size}\PY{p}{(}\PY{n}{eigenvectors}\PY{p}{,}\PY{+w}{ }\PY{l+m+mi}{1}\PY{p}{)}
\PY{+w}{        }\PY{n}{display}\PY{p}{(}\PY{n}{eigenvectors}\PY{p}{[}\PY{n}{i}\PY{p}{]}\PY{p}{)}
\PY{+w}{    }\PY{k}{end}
\PY{+w}{    }\PY{k}{return}\PY{+w}{ }\PY{n}{eigenvalues}\PY{p}{,}\PY{+w}{ }\PY{n}{eigenvectors}
\PY{k}{end}
\end{Verbatim}
\end{tcolorbox}

            \begin{tcolorbox}[breakable, size=fbox, boxrule=.5pt, pad at break*=1mm, opacityfill=0]
\prompt{Out}{outcolor}{156}{\boxspacing}
\begin{Verbatim}[commandchars=\\\{\}]
deflation\_example (generic function with 1 method)
\end{Verbatim}
\end{tcolorbox}
        
    \begin{tcolorbox}[breakable, size=fbox, boxrule=1pt, pad at break*=1mm,colback=cellbackground, colframe=cellborder]
\prompt{In}{incolor}{157}{\boxspacing}
\begin{Verbatim}[commandchars=\\\{\}]
\PY{n}{deflation\PYZus{}example}\PY{p}{(}\PY{n}{A\PYZus{}Deflation}\PY{p}{,}\PY{+w}{ }\PY{l+m+mi}{5}\PY{p}{)}
\end{Verbatim}
\end{tcolorbox}

    \begin{Verbatim}[commandchars=\\\{\}]
The new matrix is
    \end{Verbatim}

    
    \begin{Verbatim}[commandchars=\\\{\}]
3×3 Matrix\{Float64\}:
 -0.107028  -0.953945   1.18611
 -0.953945   2.80025   -1.12183
  1.18611   -1.12183   -1.04667
    \end{Verbatim}

    
    \begin{Verbatim}[commandchars=\\\{\}]
The new matrix is
    \end{Verbatim}

    
    \begin{Verbatim}[commandchars=\\\{\}]
3×3 Matrix\{Float64\}:
 -0.149139  -0.680337   0.981404
 -0.680337   1.02252    0.208205
  0.981404   0.208205  -2.04176
    \end{Verbatim}

    
    \begin{Verbatim}[commandchars=\\\{\}]
The new matrix is
    \end{Verbatim}

    
    \begin{Verbatim}[commandchars=\\\{\}]
3×3 Matrix\{Float64\}:
  0.272566   -0.539916   0.0513521
 -0.539916    1.06927   -0.101489
  0.0513521  -0.101489   0.00943407
    \end{Verbatim}

    
    \begin{Verbatim}[commandchars=\\\{\}]
The list of eigenvalues is
    \end{Verbatim}

    
    \begin{Verbatim}[commandchars=\\\{\}]
11.353452085530611
    \end{Verbatim}

    
    
    \begin{Verbatim}[commandchars=\\\{\}]
2.8149303216918904
    \end{Verbatim}

    
    
    \begin{Verbatim}[commandchars=\\\{\}]
-2.5196568848768055
    \end{Verbatim}

    
    \begin{Verbatim}[commandchars=\\\{\}]
The list of eigenvectors is
    \end{Verbatim}

    
    \begin{Verbatim}[commandchars=\\\{\}]
3-element Vector\{Float64\}:
 0.7911886727089043
 0.4401721932271454
 0.42458088097172625
    \end{Verbatim}

    
    
    \begin{Verbatim}[commandchars=\\\{\}]
3-element Vector\{Float64\}:
 -0.12230960774710155
  0.7946934140708866
 -0.5945609619585772
    \end{Verbatim}

    
    
    \begin{Verbatim}[commandchars=\\\{\}]
3-element Vector\{Float64\}:
  0.40910363121697507
  0.13622557702085586
 -0.9022620523386873
    \end{Verbatim}

    
            \begin{tcolorbox}[breakable, size=fbox, boxrule=.5pt, pad at break*=1mm, opacityfill=0]
\prompt{Out}{outcolor}{157}{\boxspacing}
\begin{Verbatim}[commandchars=\\\{\}]
(Any[11.353452085530611, 2.8149303216918904, -2.5196568848768055],
Any[[0.7911886727089043, 0.4401721932271454, 0.42458088097172625],
[-0.12230960774710155, 0.7946934140708866, -0.5945609619585772],
[0.40910363121697507, 0.13622557702085586, -0.9022620523386873]])
\end{Verbatim}
\end{tcolorbox}
        

    % Add a bibliography block to the postdoc
    
    
    
\end{document}
